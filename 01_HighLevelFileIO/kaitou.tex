\documentclass[a4j,dvipdfmx]{jarticle}
%----------------------------------------------------------------------
\usepackage{graphicx}
\usepackage{amsmath}
\usepackage{amssymb}
\usepackage{bm}
\usepackage{fancybox}
\usepackage{fancyhdr}
\usepackage{lastpage}
\usepackage{color}
\usepackage{multicol}
\usepackage{listings,jlisting}
%----------------------------------------------------------------------
\setlength{\topmargin}{-0.5in}
%\addtolength{\headheight}{1cm}
%%\setlength{\headsep}{0mm}
%\setlength{\oddsidemargin}{-0.5in}
%%\setlength{\evensidemargin}{-0.5in}
%\addtolength{\textwidth}{1.5in}
\addtolength{\textheight}{1in}
%----------------------------------------------------------------------
%\setlength{\columnsep}{2zw}
%\setlength{\columnseprule}{0.4pt}
%----------------------------------------------------------------------
\pagestyle{fancy}
\lhead{2016/04/12}
\rhead{(\thepage / \pageref{LastPage})}
\cfoot{}
\chead{\textgt{システムプログラミン2グ 課題1解答}}
%----------------------------------------------------------------------
\begin{document}
\def\lstlistingname{リスト}
\lstset{language=C,
%  numbers=left,
  basicstyle={\small\ttfamily},
%  basicstyle={\ttfamily},
  columns=[l]{fullflexible},
  keepspaces=true,
%  showspaces=true,
  frame=shadowbox
}
\lstinputlisting[caption=mycp(解答例1:真っ先に思いつく方法)]{mycp1.c}

\begin{lstlisting}[caption=実行例]
$ mycp                                     <-- コマンド行引数がない場合
Usage: mycp <srcfile> <dstfile>
$ mycp a.txt                               <-- コマンド行引数が一つしかない場合
Usage: mycp <srcfile> <dstfile>
$ mycp z.txt a.txt                         <-- コピー元が存在しない場合
z.txt: No such file or directory
$ mycp a.txt /a.txt                        <-- コピー先が書き込み禁止の場合
/a.txt: Permission denied
$ echo aaa bbb > a.txt                     <-- a.txt を作って
$ mycp a.txt b.txt                         <-- b.txt にコピーしてみる
$ cat b.txt                                <-- b.txt の内容を確認
aaa bbb
$ echo ccc ddd > c.txt                     <-- c.txt を作って
$ mycp c.txt b.txt                         <-- b.txt に上書きしてみる
$ cat b.txt                                <-- b.txt の内容を確認
ccc ddd
$
\end{lstlisting}

\newpage

\lstinputlisting[caption=mycp(解答例2:エラー処理を関数化)]{mycp2.c}

\begin{multicols}{2}
\begin{lstlisting}[caption=解答例2の改悪バージョン1)]
// インデントが余計に深くなる
int main(int argc, char *argv[]) {
  if (argc != 3) usage();
  else {
    FILE *fps = eOpen(argv[1], "rb");
    FILE *fpd = eOpen(argv[2], "wb");

    fileCopy(fps, fpd);

    fclose(fps);
    fclose(fpd);
  }
  
  return 0;
}
\end{lstlisting}

\begin{lstlisting}[caption=解答例2の改悪バージョン2]
// その上、エラー処理の存在に気付き難い
int main(int argc, char *argv[]) {
  if (argc == 3) {
    FILE *fps = eOpen(argv[1], "rb");
    FILE *fpd = eOpen(argv[2], "wb");

    fileCopy(fps, fpd);

    fclose(fps);
    fclose(fpd);
  }
  else usage();     // この行が目立たない

  return 0;
}
\end{lstlisting}
\end{multicols}

\newpage

%\lstinputlisting[caption=mydiff の解答例(その1)]{mydiff2.c}
\begin{lstlisting}[caption=mydiff の解答例(その1)]
// mydiff.c : 異なる最初の行を表示する
#include <stdio.h>
#include <stdlib.h>
#include <string.h>
#define MAXLINE 100

// 使用方法を表示して終了する関数
void usage(char *cmd) { ... }

// エラーチェック付きの拡張 fopen
FILE *eOpen(char *fname, char *mode) { ... }

// 異なる行を表示する
void printDiffLine(int l, char *line1, char *line2) {
  printf("%d行\n", l);
  printf("> %s", line1);
  printf("---\n");
  printf("> %s", line2);
  exit(1);
}

int main(int argc, char *argv[]) {
  char buf1[MAXLINE];
  char buf2[MAXLINE];

  if (argc!=3) usage(argv[0]);         // エラーになる場合を早めに判断して排除

  FILE *fp1 = eOpen(argv[1], "r");
  FILE *fp2 = eOpen(argv[2], "r");

  for (int ln=1; fgets(buf1, MAXLINE, fp1)!=NULL && 
                 fgets(buf2, MAXLINE, fp2)!=NULL;   ln++) {

    if (strcmp(buf1, buf2)!=0)
      printDiffLine(ln, buf1, buf2);
  }

  // ファイルは自動的にクローズされる
  return 0;
}
\end{lstlisting}


\begin{multicols}{2}
\begin{lstlisting}[caption=実行例]
$ cat a.txt
hello
world
!!
$ cat b.txt
hello
World
!!
$ cat c.txt
hello
world
!
$ cat d.txt
hello
world
$ cat e.txt
hello
world
!!
$ mydiff a.txt
Usage: mydiff <file1> <file2>
$ mydiff a.txt b.txt
2行
> world
---
> World
$ mydiff a.txt c.txt
3行
> !!
---
> !
$ mydiff2 a.txt d.txt
$
\end{lstlisting}
\end{multicols}

\newpage

%\lstinputlisting[caption=mydiff の解答例(その2)]{mydiff.c}
\begin{lstlisting}[caption=mydiff の解答例(その2)]
// mydiff.c : 異なる最初の行を表示する
#include <stdio.h>
#include <stdlib.h>
#include <string.h>
#define MAXLINE 100

// 使用方法を表示して終了する関数
void usage(char *cmd) { ... }

// エラーチェック付きの拡張 fopen
FILE *eOpen(char *fname, char *mode) { ... }

// 異なる行を表示する
void printDiffLine(int l, char *line1, char *line2) { ... }

// 短いファイルを表示する
void shortFile(char *fname) {
  printf("%s が短い\n", fname);
  exit(1);
}

int main(int argc, char *argv[]) {
  if (argc!=3) usage(argv[0]);

  FILE *fp1 = eOpen(argv[1], "r");
  FILE *fp2 = eOpen(argv[2], "r");

  for (int ln=1; ; ln++) {
    char buf1[MAXLINE];                      // 変数の有効範囲は
    char buf2[MAXLINE];                      //   なるべく狭くする

    char *r1 = fgets(buf1, MAXLINE, fp1);
    char *r2 = fgets(buf2, MAXLINE, fp2);

    if (r1==NULL && r2==NULL) break;
    if (r1==NULL) shortFile(argv[1]);
    if (r2==NULL) shortFile(argv[2]);

    if (strcmp(buf1, buf2)!=0)
      printDiffLine(ln, buf1, buf2);
  }

  return 0;
}

/* 実行例
$ mydiff a.txt d.txt
d.txt が短い
$ mydiff a.txt e.txt
$
*/
\end{lstlisting}

\end{document}

