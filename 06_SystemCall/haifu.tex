\documentclass[a4j,dvipdfmx]{jarticle}
%----------------------------------------------------------------------
\usepackage{graphicx}
\usepackage{amsmath}
\usepackage{amssymb}
\usepackage{bm}
\usepackage{fancybox}
\usepackage{fancyhdr}
\usepackage{lastpage}
\usepackage{color}
\usepackage{multicol}
\usepackage{listings,jlisting}
%----------------------------------------------------------------------
\setlength{\topmargin}{-0.5in}
%\addtolength{\headheight}{1cm}
%%\setlength{\headsep}{0mm}
%\setlength{\oddsidemargin}{-0.5in}
%%\setlength{\evensidemargin}{-0.5in}
%\addtolength{\textwidth}{1.5in}
\addtolength{\textheight}{1in}
%----------------------------------------------------------------------
%\setlength{\columnsep}{2zw}
%\setlength{\columnseprule}{0.4pt}
%----------------------------------------------------------------------
\pagestyle{fancy}
\lhead{2016/05/17}
\rhead{配布資料(\thepage / \pageref{LastPage})}
\cfoot{}
\chead{\textgt{システムプログラミング2 第6回}}
%----------------------------------------------------------------------
\begin{document}
\def\lstlistingname{リスト}
\lstset{language=C,
  numbers=left,
  basicstyle={\small\ttfamily},
  columns=[l]{fullflexible},
  keepspaces=true,
  frame=shadowbox,
  commentstyle=\slshape
}
%----------------------------------------------------------------------

%\begin{figure}[hbtp]
%\begin{center}
%\includegraphics[height=2.5cm]{state.pdf}
%\caption{単語の長さをカウントするアルゴリズム}
%\end{center}
%\end{figure}

今回は、ファイル操作のシステムコールについて学ぶ。
つまり、前回コマンド操作で行ったリンクの作成等をプログラム中から行う方法を学ぶ。

\begin{enumerate}
\item ファイルの削除({\tt unlink}システムコール)\\
{\tt rm}コマンドは、このシステムコールを利用している。
「ファイルの削除」は正確にはリンク(名前)の削除の意味である。
ファイルは一つもリンクを持たなくなった時に削除される。
\begin{lstlisting}[numbers=none]
書式:
         #include <unistd.h>
         int unlink(char *path);
           path:削除するリンク(ファイル)
           返り値:正常=0、エラー=-1

使用例:
         // ファイルの削除
         if (unlink("a.txt")<0) {   // "a.txt" 削除
           perror("a.txt");         // エラー原因表示
           exit(1);                 // エラー終了
         }

\end{lstlisting}

\item ディレクトリの作成({\tt mkdir}システムコール)\\
{\tt mkdir}コマンドは、このシステムコールを利用している。
第2引数で新ディレクトリのモード(rwx)を指定する。
\begin{lstlisting}[numbers=none]
書式:
         #include <sys/stat.h>
         int mkdir(char *path, int mode);
           path:新規作成するディレクトリ
           mode:新しいディレクトリの rwxrwxrwx
           返り値:正常=0、エラー=-1

使用例:
         // ディレクトリの作成
         if (mkdir("newdir", 0755)<0) {   // "newdir" を rwxr-xr-x で作成
           perror("newdir");              // エラー原因表示
           exit(1);                       // エラー終了
         }

\end{lstlisting}

\item ディレクトリの削除({\tt rmdir}システムコール)\\
{\tt rmdir}コマンドは、このシステムコールを利用している。
空ではないディレクトリを削除することはできない。
\begin{lstlisting}[numbers=none]
書式:
         #include <unistd.h>
         int rmdir(char *path);
           path:削除するディレクトリ
           返り値:正常=0、エラー=-1

使用例:
         // ディレクトリの削除
         if (rmdir("newdir")<0) {  // "newdir" 削除
           perror("newdir");       // エラー原因表示
           exit(1);                // エラー終了
         }

\end{lstlisting}

\newpage

\item リンクの作成({\tt link}システムコール)\\
{\tt ln}コマンドは、ハードリンクを作るとき、このシステムコールを利用している。

\begin{lstlisting}[numbers=none]
書式:
         #include <unistd.h>
         int link(char *oldpath, char *newpath);
           oldpath:もとの名前
           newpath:新しい名前
           返り値:正常=0、エラー=-1

使用例:
         // ハードリンクの作成
         if (link("a.txt", "b.txt")<0) { // リンク"b.txt"を作る
           perror("link");               // "a.txt"と"b.txt"のどちらが原因か不明なので
           exit(1);                      // エラー終了
         }

\end{lstlisting}

ファイルの移動(ファイル名の変更)に応用できる。
\begin{lstlisting}[numbers=none]
         unlink("b.txt");                // 念のため"b.txt"を消す(エラーは無視)
         if (link("a.txt", "b.txt")<0) { // リンク"b.txt"を作る
           ... エラー処理 ... 
         if (unlink("a.txt")<0) {        // リンク"a.txt"を消す。
           ... エラー処理 ... 

\end{lstlisting}

\item シンボリックリンクの作成({\tt symlink}システムコール)\\
{\tt ln}コマンドは、シンボリックリンクを作るとき({\tt -s}オプション使用時)、
このシステムコールを利用している。

\begin{lstlisting}[numbers=none]
書式:
         #include <unistd.h>
         int symlink(char *path, char *newpath);
           path:シンボリックリンクに書き込む内容
           newpath:新しい名前(シンボリックリンクの名前)
           返り値:正常=0、エラー=-1

使用例:
         // シンボリックリンクの作成
         if (symlink("a.txt", "b.txt")<0) { // リンク"b.txt"を作る
           perror("b.txt");                 // エラー原因は必ず"b.txt"
           exit(1);                         // エラー終了
         }

\end{lstlisting}

\item ファイルの移動(ファイル名の変更)({\tt rename}システムコール)\\
{\tt mv}コマンドは、このシステムコールを利用している。

\begin{lstlisting}[numbers=none]
書式:
         #include <stdio.h>
         int rename(char *from, char *to);
           from:もとのファイル名(パス)
           to:移動後のファイル名(パス)
           返り値:正常=0、エラー=-1

使用例:
         // ファイルの移動
         if (rename("a.txt", "b.txt")<0) { // "a.txt" を "b.txt" に変更
           perror("rename");               // エラー原因がどっちのパスか不明
           exit(1);                        // エラー終了
         }

\end{lstlisting}

\newpage
\item ファイルモードの変更({\tt chmod}、{\tt lchmod}システムコール)\\
{\tt chmod}コマンドは、このシステムコールを利用している。

\begin{lstlisting}[numbers=none]
書式:
         #include <sys/stat.h>
         int chmod(char *path, int mode);
         int lchmod(char *path, int mode);
           path:ファイル名(パス)
           mode:モード
           返り値:正常=0、エラー=-1
          (lchmodはシンボリックリンクを辿らない)

使用例:
         // ファイルモードの変更
         if (chmod("a.txt", 0644)<0) { // ファイル"a.txt"のモードを"rw-r--r--"に変更
           perror("a.txt");            // エラー原因を表示
           exit(1);                    // エラー終了
         }

\end{lstlisting}

\item シンボリックリンクの内容を読む({\tt readlink}システムコール)\\
{\tt ls}コマンドは、このシステムコールを利用している。

\begin{lstlisting}[numbers=none]
書式:
         #include <unistd.h>
         int readlink(char *path, char *buf, int size);
           path:シンボリックリンクのパス
           buf:内容を読み出す領域(バッファ)
           size:領域(バッファ)のバイト単位のサイズ
           返り値:読みだした文字数、エラー=-1

使用例:
         // シンボリックリンクの読み出し
         char *name = "b.txt";
         char buf[100];
         int n = readlink(name, buf, 99); // シンボリックリンクの内容をbufに読み出す
         if (n<0) {                       // エラーチェック
           perror(name);                  // エラー原因を表示
           exit(1);                       // エラー終了
         }
         buf[n]='\0';                     // C言語型の文字列として完成させる
         printf("%s -> %s\n", name, buf); // ls -s 表示をまねて出力

\end{lstlisting}

\item 宿題

\begin{enumerate}
\item 簡易版UNIXコマンドの作成

UNIXコマンド、{\tt rm}、{\tt mkdir}、{\tt rmdir}、
{\tt ln}、{\tt mv}の中から3つ以上を選択する。
各コマンドについて
上記のシステムコールを使用して簡易版を作ってみる。
但し、使用方法のエラーチェック、システムコールのエラーチェックは行うこと。

また、自分が作った簡易版と本物の機能の違いを調べる。

(\verb/$ man 1 rm/ 等でマニュアルを参照できる)

\item {\tt rename}システムコールの必要性

UNIXは Simple is best の思想で作られてきた。
そのため、専用の機能を用意しなくても既存の機能を組み合わせて実現できる場合は
専用機能の追加はしないことが多い。
{\tt rename}システムコールの機能は{\tt link}、{\tt unlink}システムコールの
組み合わせでも実現できそうだが{\tt rename}システムコールが追加された。
なぜ追加されたか考えなさい。
\end{enumerate}
\end{enumerate}
\end{document}
