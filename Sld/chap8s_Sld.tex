%\documentclass[dvipdfmx]{beamer}      % platex の場合
\documentclass{beamer}                 % lualatex の場合
\usepackage{mySld}

\begin{document}
\title{オペレーティングシステムの機能を使ってみよう\\
第8章 環境変数}
\date{}

%===============================================================
\begin{frame}
  \titlepage
\end{frame}

\section{環境変数}
%===============================================================
\begin{frame}[fragile]
  \frametitle{環境変数}
  \fig{scale=0.56}{environmentVariableCopy-crop.pdf}
  \begin{itemize}
  \item シェルが管理する変数
  \item シェルからプログラムにコピーされる.
  \item プログラムは実行時に環境変数の値を調べることができる.
  \item 同じプログラムで複数の言語に対応すること等ができる.
  \end{itemize}
\end{frame}

%===============================================================
\begin{frame}[fragile]
  \frametitle{環境変数と使用例(1)}
  macOSやUNIXでよく使用される環境変数
  \lst{numbers=none}{environmentVariableSample.txt}
  \begin{itemize}
  \item 本当はもっとたくさんの環境変数がある.
  \item ここでは「名前=値」形式で一覧を表示している.
  \item 次頁は\texttt{LC\_TIME}環境変数と\texttt{TZ}環境変数を変更して
    実行した例
  \end{itemize}
\end{frame}

%===============================================================
\begin{frame}[fragile]
  \frametitle{環境変数と使用例(2)}
  \lst{numbers=none}{environmentVariableLCTIME.txt}
  \begin{itemize}
  \item \texttt{LC\_TIME}環境変数は日時の表示形式を決める.
  \item \texttt{TZ}環境変数はどの地域の時刻を表示するか決める.
  \end{itemize}
\end{frame}

%===============================================================
\begin{frame}[fragile]
  \frametitle{環境変数を誰が決めるか}
  \begin{enumerate}
  \item[(1)] システム管理者\\
    システム管理者はユーザがログインした時の初期状態を決める.\\
    UNIXやmacOSでは管理者が作成したスクリプトが初期化を行う.\\
    管理者は全ユーザに共通の初期化処理をここ(\|/etc/zprofile|)に書いておく.
  \item[(2)] ユーザの設定ファイル \\
    ユーザは自分のホームディレクトリのファイルに初期化手順を書く.\\
    初期化スクリプト(\|.zprofile|)の例を示す.
    \lst{numbers=none}{dotBashRc.txt}
  \item[(3)] ユーザによるコマンド操作 \\
    シェルのコマンド操作で環境変数を操作することができる.\\
    影響範囲は操作したウインドのシェルのみである.\\
    次回のログイン時には操作結果の影響は残らない.\\
  \end{enumerate}
\end{frame}

%===============================================================
\begin{frame}[fragile]
  \frametitle{環境変数の操作(1)}
  環境変数を\emph{表示}するコマンド(printenv)
\begin{description}
\item[書式] \texttt{name}は環境変数の名前である.
\begin{lstlisting}[numbers=none]
  printenv [name]
\end{lstlisting}
\item[解説]
  \texttt{name} を省略した場合は,
  全ての環境変数の名前と値を表示する.
  \texttt{name} を書いた場合は該当のする環境変数の\emph{値だけ}表示する.
  該当する環境変数が無い場合は何も表示しない.
\item [実行例]  macOS 上での printenvコマンドの実行例を示す.
  環境変数の名前を省略して実行した場合は,
  全ての環境変数について「名前=値」形式で表示される.
  \lst{numbers=none}{environmentVariablePrintenv.txt}
\end{description}
\end{frame}

%===============================================================
\begin{frame}[fragile]
  \frametitle{環境変数の操作(2)}
  環境変数を\emph{新規作成}する手順(その1)-- shの場合 --
\begin{description}
\item[書式] 次の2ステップで操作を行う.
\begin{lstlisting}
  name=value
  export name
\end{lstlisting}
\item[解説]
  \emph{1行}で,一旦,シェル変数を作る.\\
  \emph{2行}でシェル変数を環境変数に変更する.
\item[実行例]
  \emph{1行}は\texttt{MYNAME}環境変数が存在するか確認している.\\
  (\texttt{MYNAME}環境変数は存在しないので何も表示されない.)\\
  \emph{2,3行}で値が\texttt{sigemura}の\texttt{MYNAME}環境変数を作った.\\
  \emph{4行}で\texttt{MYNAME}環境変数を確認する.\\
  (値が\texttt{sigemura}になっていることが分かる.)
  \lst{numbers=left}{environmentVariableSh.txt}
\end{description}
\end{frame}

%===============================================================
\begin{frame}[fragile]
  \frametitle{環境変数の操作(3)}
  環境変数を\emph{新規作成}する手順(その2)-- zshの場合 --
\begin{description}
\item[書式]
  次の1ステップで環境変数を作ることができる.
\begin{lstlisting}[numbers=none]
  export name=value
\end{lstlisting}
\item[解説]
  一旦,シェル変数を作ることなく環境変数を作ることができる.
\item[実行例]
  次のように動作確認ができる.
  \lst{numbers=none}{environmentVariableBash.txt}
\end{description}
\end{frame}

%===============================================================
\begin{frame}[fragile]
  \frametitle{環境変数の操作(4)}
  環境変数の値を\emph{変更}する手順
\begin{description}
\item [書式]
  \texttt{name}は環境変数の名前,\texttt{value}は新しい値である.
\begin{lstlisting}[numbers=none]
  name=value
\end{lstlisting}
\item [解説]
  「環境変数の変更」と「シェル変数の作成」は書式だけでは区別が付かない.
  変数名を間違った場合,
  間違った名前で新しいシェル変数が作成されエラーにならないので
  注意が必要である.
\item [実行例]
  \texttt{MYNAME}環境変数が既に存在している場合の実行例を示す.
  \lst{numbers=none}{environmentVariableOverWrite.txt}
\end{description}
\end{frame}

%===============================================================
\begin{frame}[fragile]
  \frametitle{環境変数の操作(5)}
  環境変数の値を\emph{参照}する手順(1)
\begin{description}
\item [書式]
  \texttt{name}は環境変数の名前である.
\begin{lstlisting}[numbers=none]
  $name
\end{lstlisting} %$
\item [解説]
  \texttt{\$name}は変数の値に置き換えられる.
\item [実行例1]
  \texttt{PATH}環境変数の値にディレクトリを追加する例.
  \lst{numbers=none}{environmentVariableReference.txt}
\end{description}
\end{frame}

%===============================================================
\begin{frame}[fragile]
  \frametitle{環境変数の操作(6)}
  環境変数の値を\emph{参照}する手順(2)
\begin{description}
\item [実行例2]
  環境変数\texttt{i}の値をインクリメントする例.\\
  \lst{numbers=none}{environmentVariableIncrement.txt}
\end{description}
  \begin{itemize}
    \item  \texttt{expr}は式の計算結果を表示するコマンド.
    \item バッククオートの内部は実行結果と置き換わる.
    \item  \texttt{printenv i}の代わりに\texttt{echo \$i}でも値を表示できる.
  \end{itemize}
\end{frame}

%===============================================================
\begin{frame}[fragile]
  \frametitle{環境変数の操作(7)}
  環境変数を\emph{削除}する手順
\begin{description}
\item [書式]
  \texttt{name}は変数の名前である.
\begin{lstlisting}[numbers=none]
  unset name
\end{lstlisting}
\item [解説]
  存在しない変数を\texttt{unset}してもエラーにならない.\\
  変数名を間違ってもエラーにならないので注意が必要である.
\item [実行例]
  \texttt{MYNAME}環境変数が既に存在している場合の実行例を示す.
  \lst{numbers=none}{environmentVariableUnset.txt}
\end{description}
\end{frame}

%===============================================================
\begin{frame}[fragile]
  \frametitle{環境変数の操作(8)}
  envコマンドを用いて環境変数を\emph{一時的に変更}する手順
\begin{description}
\item [書式]
  変数へ値を代入が続いた後にコマンドが続く.
\begin{lstlisting}[numbers=none]
  env name1=value1 name2=value2 ... command
\end{lstlisting}
\item [解説]
  最初の代入形式を環境変数の変更(作成)指示とみなす.\\
  代入形式ではないもの以降を実行すべきコマンドとみなす.
\item [実行例]
  ロケールとタイムゾーンを変更してdateを実行する.\\
  \texttt{LC\_TIME}環境変数は日時表示用のロケールを格納する.\\
  \texttt{TZ}変数はタイムゾーンを格納する.
  \lst{numbers=none}{environmentVariableEnv.txt}
\end{description}
\end{frame}

%===============================================================
\begin{frame}[fragile]
  \frametitle{ロケール(ユーザの言語や地域を定義する)}
  \texttt{LANG}環境変数や\texttt{LC\_TIME}環境変数に
  セットする値をロケール名と呼ぶ.
  ロケール名は次の組み合わせで表現される.
  \centerline{「言語コード」,「国名コード」,「エンコーディング」}
\vfill
\begin{itemize}
\item \emph{言語コード}はISO639で定義された2文字コードである.\\
  (日本語は"ja")
\item \emph{国名コード}はISO3166で定義された2文字コードである.\\
  (日本は"JP")
\item \emph{エンコーディング}は,使用する文字符号化方式を示す.\\
  (macOSやLinuxではUTF-8方式が使用される.)
\item 使用可能なロケールの一覧は\|locale -a|コマンドで表示できる.
\end{itemize}
\vfill
macOSで日本語を使用する場合のロケール名は次の通り.
\centerline{\texttt{ja\_JP.UTF-8 }(日本語\_日本.UTF-8)}
\end{frame}

%===============================================================
\begin{frame}[fragile]
  \frametitle{タイムゾーン(時差が同じ地域)}
  どの地域の時間で時刻を表示するかを環境変数で制御できる.
  \vfill
  \begin{itemize}
  \item 日本時間は協定世界時(UTC)と時差がマイナス9時間
  \item \texttt{TZ}環境変数にタイムゾーンを表す値をセットする.
  \item OSの内部の時刻は協定世界時(UTC)
  \item 時刻を表示する時に\texttt{TZ}を参照して現地時間に変換する.
  \item 日本時間は\|TZ=JST-9|となる.
    \begin{itemize}
      \item \|/usr/share/zoneinfo/|ディレクトリのファイル名でも指定できる.
      \item \|Cuba|ファイルが存在するので\|TZ=Cuba|と指定できる.
      \item \|Japan|ファイルも存在するので\|TZ=Japan|も指定できる.
      \item \|Asia/Tokyo|ファイルが存在するので\|TZ=Asia/Tokyo|も可.
    \end{itemize}
  \end{itemize}
  \vfill
\texttt{TZ}環境変数が定義されていない時は,
OSのインストール時に選択した標準のタイムゾーンが用いられる.
\end{frame}

%===============================================================
%\begin{frame}[fragile]
%  \frametitle{課題 No.7}
%  \begin{enumerate}
%  \item[1.] ここまでの実行例を試してみなさい.
%    \vfill
%  \item[2.]  囲み記事を参考に,
%    \texttt{LC\_TIME}環境変数や\texttt{TZ}環境変数を色々試してみる.
%    例えば,
%    「モスクワ時間,ロシア語表記」で現在時刻を表示するにはどうしたらよいか?
%  \end{enumerate}
%  \vfill
%\end{frame}

%===============================================================
\begin{frame}[fragile]
  \frametitle{環境変数の仕組み(0)}
  \emph{参考:プロセスの構造(6章で紹介したもの)}
  \fig{scale=0.5}{proc-crop.pdf}
\end{frame}

%===============================================================
\begin{frame}[fragile]
  \frametitle{環境変数の仕組み(1)}
  \emph{シェルによる管理}
  \fig{scale=0.5}{environmentVariableShell-crop.pdf}
  \begin{itemize}
  \item 環境変数はシェルプロセスのメモリ空間に記憶されている.
  \item コマンドが入力されるとシェルのインタプリタが意味を解釈する.
  \item 環境変数を操作するコマンドならメモリ空間を操作する.
  \item 環境変数を操作するコマンドは\emph{内部コマンド}
  \end{itemize}
\end{frame}

%===============================================================
\begin{frame}[fragile]
  \frametitle{環境変数の仕組み(2)}
  \emph{プロセスへのコピー}
  \fig{scale=0.6}{environmentVariableCopy-crop.pdf}
  \begin{itemize}
  \item シェルは子プロセスとして\emph{外部コマンド}を起動する.
  \item 外部コマンドの起動時に子プロセスに環境変数をコピーする.
  \item 子プロセスはコピーされた環境変数を参照・変更・削除できる.
  \end{itemize}
\end{frame}

%===============================================================
\begin{frame}[fragile]
  \frametitle{環境変数の仕組み(3)}
  \emph{変更した上でのコピー}
  \fig{scale=0.55}{environmentVariableEnv-crop.pdf}
  \begin{itemize}
  \item 他のプログラムを起動する時に環境変数をコピーする.
  \item envコマンドは他のプログラムを起動するプログラムの例.
    \begin{enumerate}
    \item[1.] envコマンドは自身の環境変数を変更する.
    \item[2.] envコマンドは指定されたプログラムを起動する.
    \item[3.] envコマンドは,この際,環境変数をコピーする.
    \end{enumerate}
  \end{itemize}
\end{frame}

%===============================================================
\begin{frame}[fragile]
  \frametitle{プログラムからの環境変数アクセス(1)}
  \emph{読み出し(envp仮引数,environ変数を用いる)}
  \begin{description}
  \item[データ構造] メモリ内で環境変数は次のようなデータ構造
    \fig{scale=0.60}{environmentVariableData-crop.pdf}
  \item[プログラム例] 全ての環境変数を\|name=val|形式で印刷する.
    \lst{numbers=left,lastline=8}{environmentVariablePrintAll.c}
  \end{description}
\end{frame}

%===============================================================
\begin{frame}[fragile]
  \frametitle{プログラムからの環境変数アクセス(2)}
  \emph{読み出し(getenv関数を用いる)}
  \begin{description}
  \item [書式] getenv関数に変数名を与えると値が返る.
\begin{lstlisting}[numbers=none]
  #include <stdlib.h>
  char *getenv(char *name);
\end{lstlisting}
  \item [プログラム例] \texttt{LANG}環境変数の値を表示する.
    \lst{numbers=left,lastline=15}{environmentVariablePrintLANG.c}
  \end{description}
\end{frame}

%===============================================================
\begin{frame}[fragile]
  \frametitle{プログラムからの環境変数アクセス(3)}
  \emph{作成と値の変更(setenv関数)}
  \begin{description}
  \item [書式] 変数名(\|name|),値(\|val|),
    フラグ(\|overwrite|)を与える.
\begin{lstlisting}[numbers=none]
  #include <stdlib.h>
  int setenv(char *name, char *val, int overwrite);
\end{lstlisting}
  \item [解説] \|overwrite=0|で上書き禁止になる.\\
    返り値は,正常時\|0|,エラー時\|-1|である.\\
    上書き禁止時,既に変数が存在するとエラーになる.
  \item [使用例] \|MYNAME|環境変数の値を\|sigemura|にする.
\begin{lstlisting}[numbers=none]
  setenv("MYNAME", "sigemura", 1);
\end{lstlisting}
    この例は上書き許可の場合.
  \end{description}
  \vfill
\end{frame}

%===============================================================
\begin{frame}[fragile]
  \frametitle{プログラムからの環境変数アクセス(4)}
  \emph{作成と値の変更(putenv関数)}
  \begin{description}
  \item [書式] \|name=val|形式の文字列(\|string|)を与える.
\begin{lstlisting}[numbers=none]
  #include <stdlib.h>
  int putenv(char *string);
\end{lstlisting}
  \item [解説] \|name=val|形式以外の文字列を与えるとエラーになる.\\
    返り値は正常時\|0|,エラー時\|-1|である.\\
    \|putenv|関数は常に上書き許可になる.
  \item [使用例] \|MYNAME|環境変数の値を\|sigemura|にする.
\begin{lstlisting}[numbers=none]
  putenv("MYNAME=sigemura");
\end{lstlisting}
    次の\|setenv|と同じ.
\begin{lstlisting}[numbers=none]
  setenv("MYNAME", "sigemura", 1);
\end{lstlisting}
  \end{description}
  \vfill
\end{frame}

%===============================================================
\begin{frame}[fragile]
  \frametitle{プログラムからの環境変数アクセス(5)}
  \emph{削除(unsetenv関数)}
  \begin{description}
  \item [書式] 削除する変数の名前(\|name|)を与える.
\begin{lstlisting}[numbers=none]
  #include <stdlib.h>
  int unsetenv(char *name);
\end{lstlisting}
  \item [解説] 名前(\|name|)を指定して環境変数を削除する.\\
    名前の変数が無いなどのエラー時\|-1|が返る.\\
    正常時は\|0|が返る.
  \item [使用例] \|MYNAME|環境変数を削除する.
\begin{lstlisting}[numbers=none]
  unsetenv("MYNAME);
\end{lstlisting}
  \end{description}
  \vfill
\end{frame}

\end{document}

%===============================================================
\begin{frame}[fragile]
  \frametitle{課題 No.8}
  \begin{enumerate}
  \item[1.] 外部コマンドprintenvの仕様を調べる \\
    オンラインマニュアル(\|man 1 printenv|)を読んだり,
    \|printenv|を実際に実行したりして,
    printenvコマンドの仕様を調べなさい.
    \vfill
  \item[2.] myprintenvプログラム\\
    外部コマンドprintenvと同様な働きをするmyprintenvプログラムを作成しなさい.
    なるべく本物と同じ動作をするように作ること.
  \end{enumerate}
  \vfill
\end{frame}

%===============================================================
%\begin{frame}[fragile]
%  \frametitle{}
%\end{frame}

\end{document}
