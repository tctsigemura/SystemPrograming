%\documentclass[dvipdfmx]{beamer}      % platex の場合
\documentclass{beamer}                 % lualatex の場合
\usepackage{mySld}

\begin{document}
\title{オペレーティングシステムの機能を使ってみよう\\
第3章 高水準入出力と低水準入出力}
\date{}

%===============================================================
\begin{frame}
  \titlepage
\end{frame}

%===============================================================
%\begin{frame}[fragile]
%  \frametitle{}
%\end{frame}

\section{高水準入出力と低水準入出力}
%===============================================================
\begin{frame}
  \frametitle{高水準入出力と低水準入出力}
  \emph{ファイルを読み書きするための機能} \\
  (API:Application Proguram Interface)
  \begin{itemize}
  \item \emph{高水準入出力(高水準I/O)} \\
    多くの高機能な関数群\\
    (\texttt{fprintf()},\texttt{fscanf()},
      \texttt{fputc()},\texttt{fgetc()},...)
  \item \emph{低水準出力(高水準I/O)} \\
    システムコールのこと \\
    少なく,かつ,シンプルなAPI \\
    (\texttt{open()},\texttt{read()},\texttt{write()},
      \texttt{lseek()},\texttt{close()})
  \end{itemize}
\end{frame}

%===============================================================
\begin{frame}
  \frametitle{高水準I/Oのデータ構造(書き込み)}
  \fig{scale=0.6}{HiVsLoWrite-crop.pdf}
  \begin{itemize}
  \item ファイルポインタ(\texttt{fp})
  \item \texttt{FILE}構造体
  \item バッファリング
  \item writeシステムコール
  \end{itemize}
\end{frame}

%===============================================================
\begin{frame}
  \frametitle{高水準I/Oのデータ構造(読み出し)}
  \fig{scale=0.6}{HiVsLoRead-crop.pdf}
  \begin{itemize}
  \item ファイルポインタ(\texttt{fp})
  \item \texttt{FILE}構造体
  \item readシステムコール
  \item バッファリング
  \end{itemize}
\end{frame}

%===============================================================
\begin{frame}
  \frametitle{標準入出力(標準入出力ストリーム)}
  \tbl{scale=1.0}{stdio.pdf}
  \fig{scale=0.60}{stdio-crop.pdf}
\end{frame}

%===============================================================
\begin{frame}
  \frametitle{ユニファイドI/O}
  \tbl{scale=1.0}{printfVsFprintf.pdf}
  \begin{itemize}
  \item \texttt{printf(...)}と\texttt{fprintf(stdout,...)}は同じ
  \item \texttt{fp}の代わりに\texttt{stdin},\texttt{stdout}等が使用できる.
  \item キーボードやディスプレイ(\emph{入出力装置})と\emph{ファイル}を
    同じ要領で操作できる.
  \item 入出力装置をファイルに統合=(\emph{ユニファイドI/O})
  \end{itemize}
\end{frame}

%===============================================================
\begin{frame}
  \frametitle{標準入力ストリーム}
  \fig{scale=0.6}{stdio-crop.pdf}
  \begin{itemize}
  \item ファイルポインタは\texttt{stdin}
  \item ファイルディスクリプタは 0 番
  \item ファイルディスクリプタ 0 は通常キーボードに接続
%  \item ファイルディスクリプタ 0 はプログラム起動前にシェルがオープン
  \item ファイルポインタとFILE構造体はプログラム起動時に初期化
  \item シェルはファイルディスクリプタ 0 をリダイレクト可能
  \end{itemize}
\end{frame}

%===============================================================
\begin{frame}
  \frametitle{標準出力ストリーム}
  \fig{scale=0.6}{stdio-crop.pdf}
  \begin{itemize}
  \item ファイルポインタは\texttt{stdout}
  \item ファイルディスクリプタは 1 番
  \item ファイルディスクリプタ 1 は通常ディスプレイに接続
%  \item ファイルディスクリプタ 1 はプログラム起動前にシェルがオープン
  \item ファイルポインタとFILE構造体はプログラム起動時に初期化
  \item シェルはファイルディスクリプタ 1 をリダイレクト可能
  \end{itemize}
\end{frame}

%===============================================================
\begin{frame}
  \frametitle{標準エラー出力ストリーム}
  \fig{scale=0.6}{stdio-crop.pdf}
  \begin{itemize}
  \item エーラメッセージ出力用のストリーム
  \item ファイルポインタは\texttt{stderr}
  \item ファイルディスクリプタは 2 番
  \item ファイルディスクリプタ 2 は通常ディスプレイに接続
%  \item ファイルディスクリプタ 2 はプログラム起動前にシェルがオープン
  \item ファイルポインタとFILE構造体はプログラム起動時に初期化
  \item シェルはファイルディスクリプタ 2 をリダイレクト可能
  \end{itemize}
\end{frame}

%===============================================================
\begin{frame}
  \frametitle{性能比較(1/2)}
  \begin{enumerate}
  \item[1] プログラムを準備する \\
    \begin{tabular}{l l l}
    \texttt{mycp}        & : &  高水準I/O版 \\
    \texttt{mycp2\_1}    & : &  低水準I/O版(バッファサイズ=1バイト)\\
    \texttt{mycp2\_1024} & : &  低水準I/O版(バッファサイズ=1,024バイト)\\
    \end{tabular}
  \item[2] 大きめのファイルを作る \\
    \lst{language=bash}{dd_10M.txt}
  \end{enumerate}
\end{frame}

%===============================================================
\begin{frame}
  \frametitle{性能比較(2/2)}
  \begin{enumerate}
  \item[3] 実行時間を測定方法
    \lst{language=bash}{mycp2_1.txt}
  \item[4] 実行時間の測定
    \tbl{scale=0.7}{mycp2_1_10M-crop.pdf}
  \end{enumerate}
\end{frame}

%===============================================================
\begin{frame}[fragile]
\frametitle{課題 No.2 : 三つのプログラムの性能比較}
上記の性能比較を実際に行う.
提出物は以下の通りとする.

\begin{enumerate}
\item[1] 三つのプログラムについて実行結果を整理したもの
\item[2] 使用したプログラムのソースコード
\item[3] 感想・考察(ソースコードの余白に記入する)
\end{enumerate}
\end{frame}

%===============================================================
%===============================================================
\end{document}
%===============================================================
%===============================================================

%===============================================================
\begin{frame}[fragile]
\frametitle{課題 No.2 の解答例(1/5) 低水準I/O(1/2)}
\lst{lastline=23,frame=tlr}{mycp2.c}
\end{frame}

%===============================================================
\begin{frame}[fragile]
\frametitle{課題 No.2 の解答例(2/5) 低水準I/O(2/2)}
\lst{firstline=25,frame=lrb}{mycp2.c}
\end{frame}

%===============================================================
\begin{frame}[fragile]
\frametitle{課題 No.2 の解答例(3/5) 高水準I/O(1/2)}
\lst{lastline=22,frame=tlr}{mycp.c}
\end{frame}

%===============================================================
\begin{frame}[fragile]
\frametitle{課題 No.2 の解答例(4/5) 高水準I/O(2/2)}
\lst{firstline=24,frame=lrb}{mycp.c}
\end{frame}

%===============================================================
\begin{frame}[fragile]
\frametitle{課題 No.2 の解答例(5/5) 実行時間のまとめ}

{\small  1バイトのwriteシステムコール使用の場合}
\tbl{scale=0.66}{mycp2_1_10M-crop.pdf}

{\small 1,024バイトのwriteシステムコール使用の場合}
\tbl{scale=0.66}{mycp2_1024_10M-crop.pdf}

{\small 高水準I/O使用の場合}
\tbl{scale=0.66}{mycp_10M-crop.pdf}

\end{frame}

\end{document}
