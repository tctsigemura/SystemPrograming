%\documentclass[dvipdfmx]{beamer}      % platex の場合
\documentclass{beamer}                 % lualatex の場合
\usepackage{mySld}

\begin{document}
\title{オペレーティングシステムの機能を使ってみよう\\第1章 システムプログラミング}
\date{}

\begin{frame}
  \titlepage
\end{frame}

%\begin{frame}
%  \frametitle
%  \tableofcontents
%\end{frame}

\section{システムプログラミング}
\begin{frame}
  \frametitle{システムプログラムとは}
  \begin{itemize}
  \item カーネル(OSの本体)
  \item ライブラリ(プログラムが使用するサブルーチン,DLL ...)
  \item ミドルウェア(DBMS,Webサーバ ...)
  \item ユーティリティ(ファイル操作,時計,シェル,システム管理 ...)
  \item プログラム開発環境(エディタ,コンパイラ,アセンブラ,リンカ,インタプリタ ...)
  \end{itemize}
  \fig{scale=0.6}{systemConstruct-crop.pdf}
\end{frame}

\begin{frame}
  \frametitle{システムプログラミングとは}
  \begin{itemize}
  \item システムプログラムを作成すること.
  \item 本講義ではユーティリティのプログラミングを行う.
  \end{itemize}
  \vfill
  \begin{quote}
    {\Large
      \begin{itembox}[l]{なぜシステムプログラミング?}
        「システムプログラミングを通して
          オペレーティングシステムの体感的な理解」をする.
        \end{itembox}
      }
      \vfil
      オペレーティングシステムを体感的に理解するために,
      オペレーティングシステムの機能を直接に使用する
      \emph{簡単なCLI版のユーティリティプログラムの作成(プログラミング)}
      を行う.
  \end{quote}
\end{frame}

\begin{frame}
  \frametitle{複数のプログラムが正しく実行されない}
  \begin{itemize}
  \item コンピュータの中では複数のプログラムが同時に作動している.
  \item 各プログラムが勝手に\emph{資源}にアクセスすると具合が悪い.
  \item 例えばディスクのどの領域をどのファイルが使用するか?\\
    各プログラムが勝手に決めると不具合が起こる.
  \end{itemize}
  \fig{scale=0.7}{noSystemcall-crop.pdf}
\end{frame}

\begin{frame}
  \frametitle{複数のプログラムが正しく実行される}
  \begin{itemize}
  \item OSの本体(\emph{カーネル})が代表して資源を管理する.
  \item 一般のプログラムはカーネルに処理を依頼し目的を達成する.
  \item 例えばファイルを作成してディスクのどの領域を使用するか?\\
    カーネルが責任を持って一貫した管理を行う.(集中管理)
  \item 他のプログラムは\emph{システムコール}を行いカーネルに処理を依頼.
  \end{itemize}
  \fig{scale=0.6}{systemcall-crop.pdf}
\end{frame}

\begin{frame}[fragile]
  \frametitle{システムコールの使用}
  \begin{itemize}
  \item C言語からシステムコールを利用することができる.
  \item UNIX(macOS)のC言語ではシステムコールと同じ名前の関数を呼び出すと
    システムコールの発行になる.
  \item macOS上でC言語を用いてシステムコールを直接に使用する
    ユーティリティプログラムを作成する(システムプログラミングを行う).
  \item OSの機能をシステムコールを通して実感する.
  \item OSが提供すべき機能を理解する.
  \end{itemize}
  \begin{quote}
\begin{lstlisting}[language=C++]
    // ディレクトリを作るユーティリティプログラム(mymkdir)の例
    int main(int argc, char *argv[]) {
      if (argc!=2) {
        // エラー処理
      }
      mkdir(argv[1]);    // ディレクトリを作るシステムコール
      return 0;
    }
\end{lstlisting}
  \end{quote}
\end{frame}

\end{document}

