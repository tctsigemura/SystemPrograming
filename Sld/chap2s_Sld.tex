%\Documentclass[dvipdfmx]{beamer}      % platex の場合
\documentclass{beamer}                 % lualatex の場合
\usepackage{mySld}

\begin{document}
\title{オペレーティングシステムの機能を使ってみよう\\第2章 ファイル入出力システムコール}
\date{}

%===============================================================
\begin{frame}
  \titlepage
\end{frame}

%===============================================================
%\begin{frame}[fragile]
%  \frametitle{}
%\end{frame}

\section{ファイル入出力システムコール}
%===============================================================
\begin{frame}
  \frametitle{ファイル入出力システムコール}
  \begin{itemize}
  \item ファイルの読み書きを行うシステムコールを勉強する.
  \item システムコールを直接に使用したプログラムを作成してみる.
  \item プログラムの作成にはC言語を用いる.
  \item システムコールを直接に使用する入出力を\emph{低水準入出力}と呼ぶ.\\
    これまで使用してきたものは\emph{高水準入出力}と呼ぶ.
  \end{itemize}
\end{frame}

%===============================================================
\begin{frame}
  \frametitle{高水準入出力と低水準入出力}
\begin{itemize}
\item C言語の入門で勉強した入出力関数は\emph{高水準入出力}関数.
\item システムコールを直接使用する入出力は\emph{低水準入出力}.
\item 高水準入出力関数は内部でシステムコールを利用.
\item 高機能・豊富な高水準と,シンプルな低水準.
\end{itemize}

\begin{center}
{\footnotesize\ttfamily\begin{tabular}{l | l}\hline\hline
高水準入出力関数 & 対応するシステムコール \\\hline
fopen()      & open システムコール \\
printf()     & write システムコール \\
putchar()    & write システムコール \\
fputs()      & write システムコール \\
fputc()      & write システムコール \\
...          & ...                     \\
scanf()      & read システムコール \\
getchar()    & read システムコール \\
fgets()      & read システムコール \\
fgetc()      & read システムコール \\
...          & ...                      \\
fclose()     & close システムコール \\
\end{tabular}}
\end{center}

\end{frame}

%===============================================================
\begin{frame}[fragile]
  \frametitle{open システムコール(書式1)}
  \begin{itemize}
  \item ファイルを開くシステムコール
  \item \texttt{fopen()}関数が使用している
  \end{itemize}

\begin{description}

\item[書式](オープンするだけの場合)
\begin{lstlisting}[language=C++]
#include <fcntl.h>
int open(const char *path, int oflag);
\end{lstlisting}

\item[解説](書式1,2共通)
\begin{itemize}
\item \texttt{fcntl.h}をインクルードする必要がある.
\item openシステムコールは正常時には\emph{ファイルディスクリプタ}
  (3以上の番号)を返す.
\footnote{stdinが0,stdoutが1,stderrが2なので3以降になる.}.
\item エラーが発生した時は\texttt{-1}を返す.
\item エラー原因は\texttt{perror()}関数で表示できる.
\end{itemize}

\end{description}
\end{frame}

%===============================================================
\begin{frame}[fragile]
\begin{description}
\item[引数]
\begin{itemize}
\item \|path|はオープンまたは作成するファイルのパス(名前)
\item \|oflag|はオープンの方法を表の記号定数を「\texttt{|}」で接続して書く.
(「\texttt{|}」は,C言語のビット毎の論理和演算子)
\end{itemize}

{\footnotesize\begin{tabular}{l | c | l}
\hline\hline
\emph{以下の一つ} & と & \emph{以下のいくつか} \\\hline
\|O_RDONLY|(読み出し用) &    & \|O_APPEND|(追記)  \\
\|O_WRONLY|(書き込み用) & +  & \|O_CREAT|(作成)   \\
\|O_RDWR|(読み書き両用) &    & \|O_TRUNC|(切詰め) \\
                      &    & ...      \\
\end{tabular}}

\item[使用例]
\begin{lstlisting}[numbers=none]
#include <fcntl.h>
...
int fdr, fdw, fda;
fdr=open("r.txt", O_RDONLY);              // 読み出し用にオープン
fdw=open("w.txt", O_WRONLY);              // 書き込み用にオープン
fda=open("a.txt", O_WRONLY|O_APPEND);     // 追記用にオープン

if (fdw<0) {                              // エラーチェック
  perror("w.txt");                        //  原因の表示
  exit(1);                                //  エラー終了
}
\end{lstlisting}
\end{description}
\end{frame}

%===============================================================
\begin{frame}[fragile]
  \frametitle{open システムコール(書式2)}
  ファイルが存在しない時はファイルを自動的に作った上で開く.

\begin{description}
\item[書式](ファイル作成もする場合)

\|oflag|に\|O_CREAT|を含む場合は,
該当ファイルが存在しないなら新規作成してからオープンする.
新規作成するファイルの\emph{保護モード}を\|mode|で指定する.

\begin{lstlisting}[language=C++]
#include <fcntl.h>
int open(const char *path, int oflag, mode_t mode);
\end{lstlisting}

\begin{itemize}
\item \|mode_t| は,16bit の整数型(16bit \|int|)である.
\item \|mode| は,作成されるファイルの保護モードである.
\item \|mode| は,8進数で記述することが多い.
\end{itemize}

{\footnotesize\ttfamily\begin{center}\begin{tabular}{l l l l}
0: --- ~~~ & 2: -w-  ~~~~ & 4: r-- ~~~ & 6: rw- \\
1: --x     & 3: -wx       & 5: r-x     & 7: rwx \\
\end{tabular}\end{center}}

\item[使用例]
\begin{lstlisting}[numbers=none]
fd=open("a.txt", O_WRONLY|O_CREAT, 0644);  // モードは rw-r--r--
\end{lstlisting}

\end{description}
\end{frame}

%===============================================================
\begin{frame}[fragile]
  \frametitle{ファイルの保護モード}
open システムコールの第3引数(\|mode|)は次のような 12bit の値である.

\begin{center}
{\ttfamily\small\begin{tabular}{|c|c|c||c|c|c||c|c|c||c|c|c|}
\multicolumn{1}{c}{11} &
\multicolumn{1}{c}{10} &
\multicolumn{1}{c}{ 9} &
\multicolumn{1}{c}{ 8} &
\multicolumn{1}{c}{ 7} &
\multicolumn{1}{c}{ 6} &
\multicolumn{1}{c}{ 5} &
\multicolumn{1}{c}{ 4} &
\multicolumn{1}{c}{ 3} &
\multicolumn{1}{c}{ 2} &
\multicolumn{1}{c}{ 1} &
\multicolumn{1}{c}{ 0} \\\hline
s  & s  & t & r & w & x & r & w & x & r & w & x \\\hline
\multicolumn{3}{c}{省略} &
\multicolumn{3}{c}{ユーザ} &
\multicolumn{3}{c}{グループ} &
\multicolumn{3}{c}{その他}
\end{tabular}}
\end{center}

\begin{itemize}
\item 最初の 3bit の意味は難しいのでここでは説明を省略する.
\item 他のビットは \|rwx| のどれかである.\|rwx| の意味は次の通り.
\end{itemize}

\begin{quote}
\begin{tabular}{c c l}
\texttt{r} & : & \textbf{r}ead 可(読み出し可能)   \\
\texttt{w} & : & \textbf{w}rite 可(書き込み可能)   \\
\texttt{x} & : & e\textbf{x}ecute 可(実行可能)
\end{tabular}
\end{quote}

例えば,第8ビットが \|1| だった. =>\\
ユーザ(ファイルの所有者)が read (読み出し)可能の意味.
%逆に \|0|なら read 不可の意味になる.
\end{frame}

%===============================================================
\begin{frame}[fragile]
ファイルのモードやユーザ(所有者),グループは次のようにして確認できる.

\begin{quote}
\begin{lstlisting}[numbers=none]
% ls -l a.txt
-rw-r--r--  1 sigemura  staff  0 Apr 11 05:53 a.txt
%
\end{lstlisting}
\end{quote}

実行結果から a.txt ファイルについて以下のことが分かる.
\begin{itemize}
\item モードの下位9ビットが \|110100100| である.
\item 所有者は sigemura である.
\item グループは staff である.
\end{itemize}

以上を総合すると a.txt ファイルについて以下のことが分かる.
\begin{itemize}
\item sigemura が読み書きができる.
\item staff グループに属するユーザは読むことだけできる.
\item その他のユーザも読むことだけできる.
\end{itemize}
\end{frame}

%===============================================================
\begin{frame}[fragile]
  \frametitle{readシステムコール(1)}
\begin{itemize}
\item 読み出し用にオープン済みのファイルからデータを読む.
\item ファイルの先頭から順に読み出す.\\
(\emph{シーケンシャルアクセス(順アクセス)})
\end{itemize}

\begin{description}
\item[書式](詳しくは\texttt{man 2 read}で調べる.)

\begin{lstlisting}
#include <unistd.h>
ssize_t read(int fildes, void *buf, size_t nbyte);
\end{lstlisting}

\item[解説]
\begin{itemize}
\item \|unistd.h|をインクルードする必要がある.
\item \|ssize_t|は 64bit int 型.
\item 正常時には読んだデータのバイト数(正の値)を返す.
\item EOF では \|0| を返す.
\item エラーが発生した時は\|-1|を返す.
\item エラーの原因は \|perror()| 関数で表示できる.
\end{itemize}

\end{description}
\end{frame}

%===============================================================
\begin{frame}[fragile]
  \frametitle{readシステムコール(2)}
\begin{description}
\item[引数]
\begin{itemize}
\item \|fildes|はオープン済みのファイルディスクリプタ
\item \|buf|はデータを読み出すバッファ領域を指すポインタ
\item \|nbyte|はバッファ領域の大きさ(バイト単位)
\end{itemize}

\item[使用例1]
\begin{itemize}
\item \|fd|は open システムコールでオープン済みと仮定
\item \|buf|はバッファ用の \|char| 型の大きさ100の配列
\item \|char|型は1バイトなので,配列全体で100バイト
\item 3回の\|read|によりファイルの先頭から順に100バイトずつ読む
\end{itemize}

\begin{lstlisting}[language=C++]
char buf[100];
n = read(fd, buf, 100); // 1回目
n = read(fd, buf, 100); // 2回目
n = read(fd, buf, 100); // 3回目
\end{lstlisting}

\end{description}
\end{frame}

%===============================================================
\begin{frame}[fragile]
  \frametitle{read システムコール(3)}
\begin{description}
\item[使用例2]
\begin{itemize}
\item ループでファイルの先頭から順にデータを読み出す例
\item \|n|の値が\|0|以下になったら EOF かエラー
\item EOFかエラーになったらループを終了
\end{itemize}

\begin{lstlisting}[numbers=none]
while ((n=read(fd, buf, 100)) > 0) { // 読む
   ... 読んだ n バイトのデータを処理する ...
}
\end{lstlisting}
\end{description}
\end{frame}

%===============================================================
\begin{frame}[fragile]
  \frametitle{writeシステムコール}
\begin{itemize}
\item 書き込み用にオープン済みのファイルへデータを書き込む.
\item ファイルの先頭から順にデータを書き込む.\\
(\emph{シーケンシャルアクセス})
\item ファイルの最後に達するまでは元々あったデータを上書きする.
\item ファイルの最後に書き込むとファイル長が延長される.
\end{itemize}

\begin{description}
\item[書式](詳しくは\texttt{man 2 write}で調べる.)
\begin{lstlisting}[numbers=none]
#include <unistd.h>
ssize_t write(int fildes, void *buf, size_t nbyte);
\end{lstlisting}

\item[解説]
\begin{itemize}
\item ファイルに実際に書き込んだデータのバイト数を返す.
\item 返された値が\|nbyte|と一致しない場合はエラー?
\end{itemize}

\item[使用例]
ファイルに \|abc| の3バイトを書き込む.
\begin{lstlisting}[numbers=none]
char *a = "abc";
n = write(fd, a, 3);       // nが3以外ならエラーが疑われる
\end{lstlisting}
\end{description}
\end{frame}

%===============================================================
\begin{frame}[fragile]
  \frametitle{lseekシステムコール(1)}
\begin{itemize}
\item オープン済みファイルの読み書き位置を移動する.
\item lseekシステムコールと組み合わせることで,
read,writeシステムコールを用いたファイルの
\emph{ランダムアクセス(直接アクセス)}が可能になる.
\end{itemize}

\begin{description}
\item[書式](詳しくは\texttt{man 2 lseek}で調べる.)
\begin{lstlisting}[numbers=none]
#include <unistd.h>
off_t lseek(int fildes, off_t offset, int whence);
\end{lstlisting}

\item[解説]
\begin{itemize}
\item \|fildes|はオープン済みのファイルディスクリプタ
\item \|off_t|型は64bit \|int| 型
\item ファイルの現在の読み書き位置を\|offset|に移動
\item \|offset|の意味は\|whence|によって変化する.
\item 正常時は新しい読み書き位置が返される.
\item エラーが発生した時は\|-1|を返す.
\item エラーの原因は \|perror()| 関数で表示できる.
\end{itemize}
\end{description}
\end{frame}

%===============================================================
\begin{frame}[fragile]
  \frametitle{lseekシステムコール(2)}
\begin{description}
\item[whence の意味]
\begin{tabular}{l | l}
\hline\hline
\|whence|    & \emph{意 味} \\\hline
\|SEEK_SET|  &  \|offset|はファイルの先頭からのバイト数  \\
\|SEEK_CUR|  &  \|offset|は現在の読み書き位置からのバイト数  \\
\|SEEK_END|  &  \|offset|はファイルの最後からのバイト数  \\
\end{tabular}

\vfill

\item[使用例]
\|fd|はオープン済みのファイルディスクリプタとする.

\begin{lstlisting}[numbers=none]
lseek(fd,  200, SEEK_SET);    // 先頭から200バイトに移動する.
lseek(fd, -100, SEEK_CUR);    // 現在地から100バイト後ろに移動する.
lseek(fd,  -10, SEEK_END);    // 最後から10バイト後ろに移動する.
\end{lstlisting}

\end{description}
\end{frame}

%===============================================================
\begin{frame}[fragile]
  \frametitle{closeシステムコール}
ファイルを閉じる.
\begin{description}
\item[書式](詳しくは\texttt{man 2 close}で調べる.)

\begin{lstlisting}[numbers=none]
#include <unistd.h>
int close(int fildes);
\end{lstlisting}

\item[解説]
\begin{itemize}
\item オープン済みのファイルを閉じる.
\item 引数はオープン済みのファイルディスクリプタ
\item 多数のファイルを開くプログラムでは不要になったものをクローズしないと,
同時に開くことができるファイル数の上限を超えることがある.
\end{itemize}

\item[使用例]
\|fd|はオープン済みのファイルディスクリプタとする.

\begin{lstlisting}[numbers=none]
close(fd);
\end{lstlisting}
\end{description}
\end{frame}

\end{document}

%===============================================================
\begin{frame}[fragile]
\frametitle{課題 No.1}
\begin{itemize}
\item mycp プログラムを作る.
\item 但し,open,read,write,close システムコールを用いる.
\item バッファサイズより大きなファイルのコピーもできること.
\item ファイルサイズがバッファサイズの整数倍とは限らない.
\item open システムコールエラーチェックは必須.
\item エラーメッセージは\|perror()|関数で表示する.
\end{itemize}

\begin{lstlisting}[numbers=none]
% dd if=/dev/urandom of=srcfile bs=1024 count=10 # ランダムな内容の
10+0 records in                                  #   10KiBのファイルを作成する
10+0 records out
10240 bytes transferred in 0.001528 secs (6701462 bytes/sec)
% ./mycp srcfile destfile                        # mycp プログラムを実行する
% cmp srcfile destfile                           # コピー元とコピー先ファイルを比較する
%                                                # 内容が同じなら何も表示されない
\end{lstlisting}
\end{frame}

%===============================================================
%===============================================================
\end{document}
%===============================================================
%===============================================================


%===============================================================
\begin{frame}[fragile]
  \frametitle{課題 No.1 の解答例(1/5)}
  \lst{language=C,lastline=13,caption=プログラム例(1/3)}{mycp2.c}
  \begin{itemize}
    \item インクルードファイル
    \item バッファサイズの定義
    \item エラー終了関数
  \end{itemize}
\end{frame}

%===============================================================
\begin{frame}[fragile]
  \frametitle{課題 No.1 の解答例(2/5)}
  \lst{language=C,firstline=14,lastline=24,caption=プログラム例(2/3)}
      {mycp2.c}
  \begin{itemize}
    \item \texttt{main()}関数
    \item 変数の宣言,バッファの宣言
    \item コマンドライン引数のチェック
  \end{itemize}
\end{frame}

%===============================================================
\begin{frame}[fragile]
  \frametitle{課題 No.1 の解答例(3/5)}
  \lst{language=C,firstline=25,caption=プログラム例(3/3)}{mycp2.c}
  \begin{itemize}
    \item \texttt{open()}のフラグに注目!!
    \item \texttt{write()}には\texttt{len}を渡すこと.
  \end{itemize}
\end{frame}

%===============================================================
\begin{frame}[fragile]
  \frametitle{課題 No.1 の解答例(4/5)}
  \lst{language=bash,lastline=15,caption=実行例(1/2)}{mycp2.txt}
  \begin{itemize}
    \item コマンドライン引数のエラーチェックの動作確認
    \item open に関する動作確認
    \item その他
  \end{itemize}
\end{frame}

%===============================================================
\begin{frame}[fragile]
  \frametitle{課題 No.1 の解答例(5/5)}
  \lst{language=bash,firstline=16,caption=実行例(2/2)}{mycp2.txt}
  \begin{itemize}
    \item read/write の繰り返しに関する動作確認
    \item ファイル長がバッファ長の倍数でない場合の動作確認
    \item ファイルが短くなる場合の動作確認
    \item その他
  \end{itemize}
\end{frame}

\end{document}
