%\documentclass[dvipdfmx]{beamer}      % platex の場合
\documentclass{beamer}                 % lualatex の場合
\usepackage{mySld}

\begin{document}
\title{オペレーティングシステムの機能を使ってみよう\\
第10章 UNIX シェル}
\date{}

%===============================================================
\begin{frame}
  \titlepage
\end{frame}

\section{UNIX シェル}
%===============================================================
\begin{frame}[fragile]
  \frametitle{UNIXのシェルとは}
  \begin{itemize}
    \item CLI(Command Line Interface)方式のコマンドインタプリタ
    \item macOSのFinderやWindowsのExplorerはGUI方式のシェル
    \item CLI版のシェルは,sh,bash,ksh,zsh,csh,tcshなど
    \item UNIXシェルの持つべき機能
      \begin{enumerate}
      \item[1.] 外部コマンド(プログラム)の起動
      \item[2.] カレントディレクトリの変更
      \item[3.] 環境変数の管理
      \item[4.] 入出力のリダイレクト,パイプ(\|<|,\|>|,\verb;|;)
      \item[5.] ジョブの管理(\|jobs|,\|fg|,\|bg|など)
      \item[6.] ファイル名の展開(ワイルドカード(\|*|,\|?|))
      \item[7.] 繰り返しや条件判断
      \item[8.] スクリプトの実行(処理の自動化)
      \end{enumerate}
  \end{itemize}
  \vfill
\end{frame}

%===============================================================
\begin{frame}[fragile]
  \frametitle{簡易UNIXシェル(\texttt{myshell)}}
  \begin{itemize}
  \item \emph{特徴}\\
    C言語で70行以内で記述できる簡易シェル
  \item \emph{できること}
    \begin{enumerate}
    \item[1.] 外部コマンド(プログラム)の起動
    \item[2.] カレントディレクトリの変更
    \end{enumerate}
  \item \emph{目的}\\
    myshell の構造を学び,
    「fork-exec 方式」,「環境変数」,「リダイレクト」等への理解を深める.
  \item \emph{目標}\\
    「環境変数の管理機能」,「リダイレクト機能」
    をmyshellに追加できるようになる.
  \end{itemize}
  \vfill
\end{frame}

%===============================================================
\begin{frame}[fragile]
  \frametitle{基本構造(\texttt{main()}関数)}
  \lst{numbers=left,xleftmargin=5mm,firstline=46,lastline=66}{myshell.c}
\end{frame}

%===============================================================
\begin{frame}[fragile]
  \frametitle{コマンド行の解析(\texttt{parse()}関数)(1)}
  \lst{numbers=left,xleftmargin=5mm,firstline=10,lastline=20}{myshell.c}
  \vfill
\end{frame}

%===============================================================
\begin{frame}[fragile]
  \frametitle{コマンド行の解析(\texttt{parse()}関数)(2)}
  \fig{scale=0.8}{myshellArgs-crop.pdf}
\end{frame}

%===============================================================
\begin{frame}[fragile]
  \frametitle{コマンドの実行(\texttt{execute()}関数)}
  \lst{numbers=left,xleftmargin=5mm,firstline=22,lastline=44}{myshell.c}
  \vfill
\end{frame}

\end{document}

%===============================================================
\begin{frame}[fragile]
  \frametitle{課題 No.11}
  \begin{enumerate}
  \item[1.] myshell に環境変数を追加するコマンド setenv ,\\
    削除するコマンド unsetenv を追加しなさい.
  \item[2.] myshell にリダイレクト機能を追加しなさい.
  \end{enumerate}

  \begin{lstlisting}[xleftmargin=5mm]
Command: setenv A B
Command: printenv A
B
Command: unsetenv A
Command: printenv A
Command: echo aaa > a.txt
Command: cat a.txt
aaa
Command:
  \end{lstlisting}
  \vfill
\end{frame}

%===============================================================
%\begin{frame}[fragile]
%  \frametitle{}
%\end{frame}

\end{document}
