%\documentclass[dvipdfmx]{beamer}      % platex の場合
\documentclass{beamer}                 % lualatex の場合
\usepackage{mySld}

\begin{document}
\title{オペレーティングシステムの機能を使ってみよう\\
第9章 プロセスの生成とプログラムの実行}
\date{}

%===============================================================
\begin{frame}
  \titlepage
\end{frame}

\section{プロセスの生成とプログラムの実行}
%===============================================================
\begin{frame}[fragile]
  \frametitle{spawn方式とfork-exec方式}
  新しいプログラムを実行する方式は次の二種類がある.
  \vfill
  \begin{itemize}
  \item \emph{spawn方式(スポーン:卵を産む方式)} \\
    Windows等で使用されてきた. \\
    次の3ステップを一つのシステムコールで行う.
    \begin{enumerate}
    \item[1.] プロセスを作る.
    \item[2.] プロセスを初期化する.
    \item[3.] プログラムを実行する.
    \end{enumerate}
  \vfill
  \item \emph{fork-exec方式(分岐-実行方式)} \\
    UNIX系のOSで使用されてきた. \\
    次の3ステップを二つのシステムコールとプログラムで行う.
    \begin{enumerate}
    \item[1.] プロセスを作る(forkシステムコール).
    \item[2.] プロセスを初期化する(ユーザプログラム).
    \item[3.] プログラムを実行する(execシステムコール).
    \end{enumerate}
  \end{itemize}
  \vfill
\end{frame}

%===============================================================
\begin{frame}[fragile]
  \frametitle{spawn方式}
  \|posix_spawn|の例 \\
  \begin{description}
  \item[書式] 次の通りである.
\begin{lstlisting}[numbers=none]
  #include <spawn.h>
  int posix_spawn(pid_t *pid, const char *path,
            const posix_spawn_file_actions_t *file_actions,
            const posix_spawnattr_t *attrp,
            char *const argv[], char *const envp[]);
\end{lstlisting}

  \item[解説]
    新しいプロセスを作り\|path|で指定したプログラムを実行する.

  \item[引数]
    \|pid|は新しいプロセスのプロセス番号を格納する変数を指すポインタ.
    \|path|は実行するプログラムを格納したファイルのパスである.
    絶対パスでも相対パスでも良い.
    \|file_actions|,\|attrp|はプロセスの初期化を指示する
    データ構造へのポインタ.
    \|argv|,\|envp|は実行されるプログラムに渡す
    コマンド行引数と環境変数である.
  \end{description}
\end{frame}

%===============================================================
\begin{frame}[fragile]
  \frametitle{fork-exec方式(1)}
  \fig{scale=0.6}{forkExec-crop.pdf}
  \begin{enumerate}
  \item[1.] 新しいプロセス(子プロセス)を作る(\emph{forkシステムコール}).
  \item[2.] ユーザプログラムに従い子プロセスが自ら初期化処理を行う.
  \item[3.] 新しいプログラムをロード・実行(\emph{execveシステムコール})する.
  \end{enumerate}
  プログラムで初期化処理を行うので柔軟性が高い.
  \vfill
\end{frame}

%===============================================================
\begin{frame}[fragile]
  \frametitle{fork-exec方式(2)}
  \emph{プログラムのロードと実行(execveシステムコール)} の概要\\
  \fig{scale=0.6}{execve-crop.pdf}
  \begin{itemize}
  \item プロセスのメモリ空間に新しいプログラムをロードする.
  \item execveシステムコールを発行したプログラムは上書きされて消える.
  \item プロセスの仮想CPUはリセットされプログラムの先頭から実行.
  \item プロセスが新しいプログラムに\emph{変身}した.
  \end{itemize}
\end{frame}

%===============================================================
\begin{frame}[fragile]
  \frametitle{fork-exec方式(3)}
  \emph{execveシステムコール} \\
  \begin{description}
  \item[書式] execveシステムコールの書式は次の通りである.
\begin{lstlisting}[numbers=none]
  #include <unistd.h>
  int execve(const char *path,
             char *const argv[], char *const envp[]);
\end{lstlisting}

  \item[解説]
    自プロセスで\|path|で指定したプログラムを実行する.
    正常時にはexecveを実行したプログラムは新しいプログラムで上書きされ消える.
    execveシステムコールが戻る(次の行が実行される)のはエラー発生時だけである.

  \item[引数]
    \|path|は実行するプログラムを格納したファイルのパスである.
    絶対パスでも相対パスでも良い.
    \|argv|,\|envp|は新しいプログラムに渡すコマンド行引数と環境変数である.
  \end{description}
  \vfill
\end{frame}

%===============================================================
\begin{frame}[fragile]
  \frametitle{fork-exec方式(4)}
  \emph{execveシステムコールの使用例1} \\
  \lst{numbers=none}{exectest1.c}
  \begin{itemize}
  \item \texttt{/bin/date}プログラムをロード・実行する.
  \item dateプログラムの\texttt{argv}配列を準備して渡す.
  \item 環境変数は自身のものを渡す.
  \item execveが戻ってきたら無条件にエラー処理をする.
  \end{itemize}
  \vfill
\end{frame}

%===============================================================
\begin{frame}[fragile]
  \frametitle{fork-exec方式(5)}
  \emph{execveシステムコールの使用例2} \\
  \lst{numbers=none}{exectest3.c}
  \begin{itemize}
  \item (環境変数を変更=初期化処理)をした上でexecveする.
  \item \texttt{putenv()}関数を用いて自身の環境変数を書き換え.
  \item execveに自身の環境変数を渡す.
  \end{itemize}
  \vfill
\end{frame}

%===============================================================
\begin{frame}[fragile]
  \frametitle{fork-exec方式(6)}
  \emph{execveシステムコールの使用例3} \\
  \lst{numbers=none}{exectest2.c}
  \begin{itemize}
  \item 全く新しい環境変数の一覧を渡す例.
  \item dateプログラムが必要とする環境変数だけの配列を渡す.
  \end{itemize}
  \vfill
\end{frame}

%===============================================================
\begin{frame}[fragile]
  \frametitle{fork-exec方式(7)}
  \emph{execveシステムコールの使用例4} \\
  \lst{numbers=none}{exectest4.c}
  \begin{itemize}
  \item 複数のコマンド行引数をもつプログラム(echo)の実行例.
  \item \texttt{argv[0]}にプログラムの名前を入れ忘れないように.
  \end{itemize}
  \vfill
\end{frame}

%===============================================================
\begin{frame}[fragile]
  \frametitle{fork-exec方式(8)}
  \emph{execveシステムコールのラッパー関数} \\
  \begin{itemize}
  \item 関数の内部でexecveシステムコールを発行(wrapper)
  \end{itemize}
  \begin{description}
  \item[書式] 四つのラッパー関数の書式をまとめて掲載
\begin{lstlisting}[numbers=none]
  #include <unistd.h>
  int execv(const char *path, char *const argv[]);
  int execvp(const char *file,  char *const argv[]);
  int execl(const char *path,
            const char *argv0, ... , *argvn, NULL);
  int execlp(const char *file,
            const char *argv0, ... , *argvn, NULL);
\end{lstlisting}
  \item[意味] \texttt{execv("/bin/date", argv);} \\
    →  \texttt{execve("/bin/date", argv, environ);} \\
    \texttt{execvp("date", argv);} \\
    →  \texttt{execve("/bin/date", argv, environ);} \\
    \texttt{execl("/bin/echo", "echo", "aaa", "bbb", NULL);} \\
    \texttt{execlp("echo", "echo", "aaa", "bbb", NULL);} \\
  \end{description}
  \vfill
\end{frame}

%===============================================================
\begin{frame}[fragile]
  \frametitle{fork-exec方式(9)}
  \emph{入出力のリダイレクト1} \\
  \begin{itemize}
  \item リダイレクトの復習
\begin{lstlisting}[numbers=none]
  % echo aaa bbb
  aaa bbb                     <--- echoの出力が表示される
  % echo aaa bbb  > a.txt     <--- echoの出力が表示されない
  % cat a.txt
  aaa bbb                     <--- echoの出力がa.txtに格納されてた
\end{lstlisting} %$
  \item プログラムは標準入出力を自らオープンする必要が無かった.
  \item プロセスの状態がexecve前のプログラムから引き継がれるから.
  \fig{scale=0.6}{execve-crop.pdf}
  \end{itemize}
  \vfill
\end{frame}

%===============================================================
\begin{frame}[fragile]
  \frametitle{fork-exec方式(10)}
  \emph{入出力のリダイレクト2} \\
  \begin{itemize}
  \item リダイレクトはプログラムのロード・実行前にシェルが行う.
  \item シェルが標準入出力をファイルに接続してからexecveしている.
  \item 原理を表すプログラム例
    \lst{numbers=none,lastline=17}{exectest5.c}
  \end{itemize}
  \vfill
\end{frame}

%===============================================================
\begin{frame}[fragile]
  \frametitle{課題 No.9}
  \begin{enumerate}
  \item[1.] 前のページ(リスト9.5)のプログラムを入力し実行してみる.
  \item[2.] 入力のリダイレクトをするプログラム例を作る. \\
    ヒント:標準入力のファイルディスクリプタは0番である.
  \item[3.] envコマンドのクローンmyenv \\
    \|putenv()|がエラーになるまでコマンド行引数を環境変数の設定と思って使う.
    残りが実行するコマンドを表している.
    下の実行例では\\
    \|putenv(argv[1]);|\\
    \|putenv(argv[2]);|\\
    \|putenv(argv[3]);|\\
    \|execvp(argv[3], &argv[3]);|\\
    (三回目の\|putenv()|はエラーになる)\\
    が実行されるようにプログラムを作る.
    \vfill
    \|% ./myenv LC_TIME=ja_JP.UTF-8 TZ=Cuba ls -l| %$
  \end{enumerate}
   \vfill
\end{frame}

%===============================================================
\begin{frame}[fragile]
  \frametitle{fork-exec方式(11)}
  \emph{新しいプロセスを作る(forkシステムコール)1} \\
  \fig{scale=0.6}{fork-crop.pdf}
  \begin{itemize}
  \item forkシステムコールはプロセスのコピー\emph{分身}を作る.
  \item もともとのプロセスが\emph{親プロセス},分身が\emph{子プロセス}.
  \item \emph{分身}はPID以外は同じ(\emph{CPUのPCも同じ}).
  \item 子プロセスはforkシステムコールの途中から実行を開始する.
  \end{itemize}
\end{frame}

%===============================================================
\begin{frame}[fragile]
  \frametitle{fork-exec方式(12)}
  \emph{新しいプロセスを作る(forkシステムコール)2} \\

\begin{description}
\item[書式] fork の書式を示す.
\begin{lstlisting}[numbers=none]
  #include <unistd.h>
  int fork(void);
\end{lstlisting}
\item[解説] 
  forkシステムコールが終了する際,
  親プロセスには子プロセスのPIDが返され,
  子プロセスには\|0|が返される.
  プログラムはこの値を目印に自分が親か子か判断できる.
  エラー時は,親プロセスに\|-1|が返され子プロセスは作られない.
\end{description}
\lst{xleftmargin=5mm,firstline=3,lastline=11,numbers=none}{forktest.c}
\end{frame}

%===============================================================
\begin{frame}[fragile]
  \frametitle{fork-exec方式(13)}
  \emph{新しいプロセスを作る(forkシステムコール)3} \\

\begin{description}
\item[使用例]
  親プロセスと子プロセスが並行実行される状態になる.
\end{description}
\lst{numbers=left, xleftmargin=5mm,lastline=17}{forktest.c}
\end{frame}

%===============================================================
\begin{frame}[fragile]
  \frametitle{fork-exec方式(14)}
  \emph{プロセスの終了と待ち合わせ}\\
  \vfill
  例えば次のような手順で処理がされる.
  \begin{itemize}
  \item 親プロセスは子プロセスをいくつか作成し,\\
    それらに同時に並行して処理を行わせる.
  \item 子プロセスは処理を終えると終了する.
  \item 子プロセスが処理を終えると,親プロセスは \\
    子プロセスが正常に終了したかチェックする.\\
    \vfill
    \verb;% ls -l | grep rwx;\hfil ←  \hfil 二つのプロセスが並列実行される
    \vfill
  \item 子プロセスが処理結果と共に自身を終了する. \\
    → \emph{exitシステムコール}
  \item 親プロセスが子プロセスの終了を待つ. \\
    → \emph{waitシステムコール}
  \end{itemize}
\end{frame}

%===============================================================
\begin{frame}[fragile]
  \frametitle{fork-exec方式(15)}
  \emph{exitシステムコール}:自プロセスを終了する.

  \begin{description}
  \item[書式] exitシステムコールの書式を示す.
\begin{lstlisting}[numbers=none]
  #include <stdlib.h>
  void exit(int status);
\end{lstlisting}

  \item[解説]
    \begin{itemize}
    \item プロセスが終了するのでexitは戻らない.
    \item \|status|はプロセスの\emph{終了ステータス}である.\\
      (\|0 <= status <= 255|)\\
    \item 親プロセスはwaitで終了ステータスを受け取る.
    \item Cプログラムのmain関数はexitの引数で実行.\\
    \vfill
    \texttt{exit(main(argc, argv, envp));}
    \vfill
    \item 次のCプログラムは同じ結果になる.\\
      \begin{minipage}{0.3\columnwidth}
\begin{lstlisting}[numbers=none]
  int main() {
    ...
    exit(1);
    ...
  }
\end{lstlisting}
      \end{minipage} ~~=~~~
      \begin{minipage}{0.3\columnwidth}
\begin{lstlisting}[numbers=none]
  int main() {
    ...
    return 1;
    ...
  }
\end{lstlisting}
      \end{minipage}
    \end{itemize}
  \end{description}
\end{frame}

%===============================================================
\begin{frame}[fragile]
  \frametitle{fork-exec方式(16)}
  \emph{waitシステムコール}:親プロセスが子プロセスの終了を待つ.

  \begin{description}
  \item[書式] waitシステムコールの書式を示す.

\begin{lstlisting}[numbers=none]
  #include <sys/wait.h>
  pid_t wait(int *status);
  WIFEXITED(status)   // 子プロセスがexitした(マクロ)
  WIFSIGNALED(status) // 子プロセスがシグナルで終了した(マクロ)
  WEXITSTATS(status)  // 子プロセスの終了ステータス(マクロ)
  WTERMSIG(status)    // 子プロセスを終了させたシグナルの番号(マクロ)
\end{lstlisting}

  \item[解説] 
    \begin{itemize}
    \item 終了したプロセスの番号(PID:正の値)を返す.
    \item エラーが発生した場合に\|-1|を返す.
    \item \|status|にプロセスの終了理由等が格納される.\\
      \|status| の内容は上記のマクロで読み取る.
    \end{itemize}
  \end{description}
\end{frame}

%===============================================================
\begin{frame}[fragile]
  \frametitle{fork-exec方式(17)}
  \emph{プログラム例}
  \lst{numbers=left,xleftmargin=5mm}{forkexec1.c}
\end{frame}

%===============================================================
\begin{frame}[fragile]
  \frametitle{fork-exec方式(18)}
  \emph{プログラムの解説}
  \fig{scale=0.6}{forkExecWaitExit-crop.pdf}
\end{frame}

%===============================================================
\begin{frame}[fragile]
  \frametitle{fork-exec方式(19)}
  \lst{numbers=left,firstnumber=5,xleftmargin=5mm,firstline=5,
    basicstyle={\fontsize{8pt}{8.5pt}\selectfont\ttfamily}}{forkexec2.c}
\end{frame}

%===============================================================
\begin{frame}[fragile]
  \frametitle{fork-exec方式(20)}
  \lst{numbers=left,firstnumber=5,xleftmargin=5mm,firstline=5,
    basicstyle={\fontsize{8pt}{8.5pt}\selectfont\ttfamily}}{forkexec3.c}
\end{frame}

\end{document}

%===============================================================
\begin{frame}[fragile]
  \frametitle{課題 No.10}
  \begin{enumerate}
  \item[1.] 次々リダイレクトしてdateを実行するプログラム \\
    コマンド行引数で「環境変数とファイル名」の組を複数指定し,
    環境変数を変更した上で出力をファイルにリダイレクトしdateを実行する
    プログラムを作りなさい.
    例えば次のように実行すると,現在時刻をキューバ時間で表したものが\|c.txt|に
    ローマ時間で表したものが\|r.txt|に格納される.
    
    \|% ./a.out TZ=Cuba c.txt TZ=Europe/Rome r.txt| %$

  \item[2.] system 関数のクローン mysystem \\
    \|system()| 関数の仕様を調べて,なるべく同じものを作りなさい.
    C言語中で\|system("...");|の関数呼出しをすると,
    シェアルに以下のように入力したのと同じことが起こる.
    
    \|% /bin/sh -c "..."| %$
    
  \end{enumerate}
\end{frame}

%===============================================================
%\begin{frame}[fragile]
%  \frametitle{}
%\end{frame}

\end{document}
