%\documentclass[dvipdfmx]{beamer}      % platex の場合
\documentclass{beamer}                 % lualatex の場合
\usepackage{mySld}

\begin{document}
\title{オペレーティングシステムの機能を使ってみよう\\
第6章 プロセスとジョブ}
\date{}

%===============================================================
\begin{frame}
  \titlepage
\end{frame}

%===============================================================
%\begin{frame}[fragile]
%  \frametitle{}
%\end{frame}

\section{プロセス}
%===============================================================
\begin{frame}[fragile]
  \frametitle{プロセス}
  \fig{scale=0.8}{pc-crop.pdf}

  \begin{itemize}
    \item プログラムは機械語の羅列のこと.
    \item 同じプログラムが同時に複数実行されることもある.
    \item 実行中のプログラムのインスタンスを\emph{プロセス}と呼ぶ.
  \end{itemize}

  \begin{center}
    \emph{\Large プロセス = 実行中のプログラム}
  \end{center}
\end{frame}

%===============================================================
\begin{frame}[fragile]
  \frametitle{プロセスの構造}
  \fig{scale=0.6}{proc-crop.pdf}

  \begin{itemize}
    \item プロセスの情報,CPUの状態(仮想CPU),メモリ空間(仮想メモリ)
  \end{itemize}

  \begin{center}
    \emph{\Large プロセス = 仮想コンピュータ}
  \end{center}
\end{frame}

%===============================================================
\begin{frame}[fragile]
  \frametitle{プロセス関連のUNIXコマンド}
  \emph{psコマンド(GUIにも似たアプリがある)}
  \begin{itemize}
  \item 実行中のプロセスの一覧表を表示するコマンドである.
  \item 一方のターミナルでemacsを起動し,
    もう一方のターミナルでpsコマンドを実行した例
  \item \texttt{-bash}(最近のmacOSでは\texttt{-zsh})は
    入力されたコマンドを解釈して実行するシェルである.
    \lst{numbers=none}{ps.txt}
  \item \texttt{TTY}はttyコマンドで確認できる.
    \lst{numbers=none}{tty.txt}
  \end{itemize}
\end{frame}

%===============================================================
\begin{frame}[fragile]
  \frametitle{プロセス関連のUNIXコマンド}
  \emph{psコマンドの表示内容}
  \tbl{scale=1.0}{psOut.pdf}
\end{frame}

%===============================================================
\begin{frame}[fragile]
  \frametitle{プロセス関連のUNIXコマンド}
  \emph{psコマンドのオプション}
  \tbl{scale=1.0}{psOptions.pdf}

  \emph{psコマンドの実行例(uオプション)}
  \lst{firstline=1,lastline=5,breaklines=true,breakindent=0pt}{psLong.txt}

  \begin{itemize}
  \item u オプションで詳しい表示がされた.
  \end{itemize}
\end{frame}

%===============================================================
\begin{frame}[fragile]
  \frametitle{プロセス関連のUNIXコマンド}
  \emph{psコマンドの実行例(auオプション)}
  \lst{firstline=6,lastline=13,breaklines=true,breakindent=0pt}{psLong.txt}

  \begin{itemize}
  \item a オプションで他人のプロセスまで表示された.
  \end{itemize}
\end{frame}

%===============================================================
\begin{frame}[fragile]
  \frametitle{プロセス関連のUNIXコマンド}
  \emph{psコマンドの実行例(auxオプション)}
  \lst{firstline=14,lastline=23,breaklines=true,breakindent=0pt}{psLong.txt}

  \begin{itemize}
  \item x オプションで制御端末を持たないプロセスまで表示された.
  \end{itemize}
\end{frame}

%===============================================================
\begin{frame}[fragile]
  \frametitle{プロセス関連のUNIXコマンド}
  \emph{psコマンド\texttt{STAT}表示の意}
  \tbl{scale=0.9}{psStat.pdf}

  \begin{itemize}
  \item 前の実行例の\texttt{STAT}の意味
  \end{itemize}
\end{frame}

%===============================================================
\begin{frame}[fragile]
  \frametitle{演習5-1}
  \emph{CLIでプロセス一覧を見てみよう}
\begin{itemize}
\item システム内の全プロセスをpsコマンドで表示してみる.
\item lessコマンドとパイプで接続して表示してみる.
\item どんなプロセスが存在するかゆっくり眺める. 
\end{itemize}
  (コマンド一覧は「\texttt{0610\_UNIXコマンド(psなど).pdf}」参照)
\end{frame}

%===============================================================
\begin{frame}[fragile]
  \frametitle{プロセス関連のUNIXコマンド}
  \emph{killコマンド}
  \begin{itembox}[l]{killコマンドの書式}
    \emph{書式}
    \begin{lstlisting}[frame=none]
    kill [-シグナル] PID ...
    \end{lstlisting}
    \emph{シグナル(省略時はTERMと同じ)}
    \tbl{scale=0.9}{killOptions.pdf}
  \end{itembox}
\end{frame}

%===============================================================
\begin{frame}[fragile]
  \frametitle{プロセス関連のUNIXコマンド}
  \emph{killコマンドの使用例}
  \lst{breaklines=true,breakindent=0pt}{kill.txt}
\end{frame}

%===============================================================
\begin{frame}[fragile]
  \frametitle{演習5-2}
  \emph{CLIでプロセスを操作してみよう}
\begin{itemize}
\item 前のページの操作を自分のコンピュータで試してみる.
\end{itemize}
  (コマンド一覧は「\texttt{0610\_UNIXコマンド(psなど).pdf}」参照)
\end{frame}

\section{ジョブ}
%===============================================================
\begin{frame}[fragile]
  \frametitle{ジョブ}
  \lst{}{job.txt}
  \begin{description}
  \item[フォアグラウンド・ジョブ]
    シェルがジョブの終了を待つ.
    ジョブが終了したらプロンプトが表示される.

  \item[バックグラウンド・ジョブ]
    コマンドの最後に\texttt{\&}を付けて実行する.
    シェルがジョブの終了を待たない.
    ジョブが終了していなくてもプロンプトが表示される.
    次のジョブと並列実行ができる.
  \end{description}
\end{frame}

%===============================================================
\begin{frame}[fragile]
  \frametitle{ジョブ制御}
  \lst{}{jobControl.txt}
  \begin{description}
  \item[Ctrl-C] フォアグラウンド・ジョブにINTシグナルを送る.
  \item[Ctrl-Z] フォアグラウンド・ジョブにTSTPシグナルを送る.
  \item[jobs] そのシェルが管理しているジョブの一覧を表示する.
  \item[fg,bf] バックグラウンド・フォアグラウンドの切替え.
  \end{description}
\end{frame}

%===============================================================
\begin{frame}[fragile]
  \frametitle{課題 No.5}
  \begin{enumerate}
    \item[1.] プロセス数を調べる.
    \begin{itemize}
      \item システム内の全プロセスの個数
      \item システム内の全ての自分のプロセスの個数
    \end{itemize}
    \item[2.] 配布された3分タイマープログラムを用いて以下を行う.
    \begin{itemize}
      \item 「第1のターミナル」でタイマープログラムを起動する.
      \item 「第2のターミナル」を操作し以下のことをする.
    \end{itemize}
    \begin{enumerate}
      \item[(1)] タイマープログラムのPIDとSTATを確認する.
      \item[(2)] タイマープログラムを一時停止させる.
      \item[(3)] タイマープログラムのSTATを確認する.
      \item[(4)] タイマープログラムを再開させる.
      \item[(5)] タイマープログラムを終了させる.\\
       タイマープログラムは画面をクリアして終了したか?
    \end{enumerate}
    \item[3.] タイマープログラムを起動したターミナルだけで行う.
    \begin{enumerate}
      \item[(1)] タイマープログラムを一時停止
      \item[(2)] 一時停止したタイマープログラムを再開
      \item[(3)] タイマープログラムを終了
    \end{enumerate}
  \end{enumerate}
\end{frame}

\end{document}
