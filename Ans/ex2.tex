\documentclass[a4j,twcolumn,11pt,nomag]{ltjarticle}      % lualatex の場合
\usepackage{myAns}
\chead{\textgt{システムプログラミング2 課題 No.2 解答}}

\begin{document}
\onecolumn
\section*{課題 No.2 の解答例}

高水準I/Oを使用した場合,
バッファサイズの異なる低水準I/Oを使用した場合について,
ファイルをコピーする時間を比較した.

\subsection*{実行条件}
  \begin{center}
    \begin{tabular}{l c l}
      実行したコンピュータ & : & MacBook Pro M1 2020 \\
      実行コンピュータのOS & : & macOS 11.3 (Big Sur) \\
      ファイルサイズ      & : & 12MiB(bs=1024, count=12288) \\
    \end{tabular}  
  \end{center}  

\subsection*{実行結果}
実行時間は遅い方から以下の順であった.
\begin{enumerate}
\item 1バイトのwriteシステムコール(mycp2\_1)
  \begin{center}
    \begin{tabular}{|l | r | r | r | r |} \hline
      & \multicolumn{1}{|c|}{1回目}
      & \multicolumn{1}{|c|}{2回目}
      & \multicolumn{1}{|c|}{3回目}
      & \multicolumn{1}{|c|}{平均} \\\hline\hline
      real & 20.37 & 19.77 & 19.55 & 19.90 \\\hline
      user &  1.61 &  1.61 &  1.63 &  1.62 \\\hline
      sys  & 18.67 & 18.04 & 17.85 & 18.19 \\\hline
    \end{tabular}
  \end{center}
\item 高水準I/O(mycp)
  \begin{center}
    \begin{tabular}{|l | r | r | r | r |} \hline
      & \multicolumn{1}{|c|}{1回目}
      & \multicolumn{1}{|c|}{2回目}
      & \multicolumn{1}{|c|}{3回目}
      & \multicolumn{1}{|c|}{平均} \\\hline\hline
      real &  0.44 &  0.44 &  0.44 &  0.44 \\\hline
      user &  0.42 &  0.42 &  0.42 &  0.42 \\\hline
      sys  &  0.01 &  0.01 &  0.01 &  0.01 \\\hline
    \end{tabular}
  \end{center}
\item 1,024バイトのwriteシステムコール(mycp2\_1024)
  \begin{center}
    \begin{tabular}{|l | r | r | r | r |} \hline
      & \multicolumn{1}{|c|}{1回目}
      & \multicolumn{1}{|c|}{2回目}
      & \multicolumn{1}{|c|}{3回目}
      & \multicolumn{1}{|c|}{平均} \\\hline\hline
      real &  0.05 &  0.05 &  0.05 &  0.05 \\\hline
      user &  0.00 &  0.00 &  0.00 &  0.00 \\\hline
      sys  &  0.04 &  0.04 &  0.04 &  0.04 \\\hline
    \end{tabular}
  \end{center}
\end{enumerate}
\subsection*{考察}
\begin{enumerate}
  \item mycp2\_1とmycp2\_1024の比較
  \begin{itemize}
   \item システム時間 \\
   mycp2\_1はmycp2\_1024よりシステム時間が約400倍になっている.
   システム時間は,システムコールの処理のためにカーネルが費やした時間である.
   2つのプログラムで\|open()|,\|close()|システムコールの使い方は同じなので,
   システム時間の変化は\|read()|,\|write()|システムコールの
   実行回数の変化が原因だと考えられる.

   \|read()|,\|write()|
   システムコールの呼び出し回数はmycp\_1の方が約1000倍になっているはずなので,
   システムコール1回当たりの処理時間は0.4倍程度と考えられる.
   システムコール1回当たり読み書きするデータの量が大きく変化(1000倍)しても,
   システムコール1回当たりの処理時間はあまり(2.5倍程度)変化しない.

   \item ユーザ時間 \\
   mycp2\_1024ではユーザ時間が短く測定できなかった.
   mycp2\_1ではループを実行する回数が約1000倍なので,
   while文の実行で費やす時間,
   \|read()|,\|write()|がシステムコールを発行するための準備などで費やす時間など,
   カーネル以外が費やす時間も長くなり,ユーザ時間を測定できたものと考えられる.
  \end{itemize}

  \item mycpとmycp2\_1024の比較
  \begin{itemize}
   \item システム時間 \\
   mycpはmycp2\_1024と比較してシステム時間が短くなっている.
   mycpのシステムコール発行回数はmycp2\_1024より少ないと推察できる.
   このことから高水準I/Oが用いるバッファは1024バイトより大きく,
   mycpのシステムコール呼び出し回数を
   mycp2\_1024より少なくしていると考えられる。

   \item ユーザ時間 \\
   mycpもループの実行回数が多いためwhile文や\|getc()|,\|putc()|の実行回数が多く,
   計測できる程度のユーザ時間を費やしたと考えられる.
   \|getc()|,\|putc()|は
   FILE構造体のバッファを操作する処理を1バイト毎に行っており,
   この処理時間がユーザ時間に算入される.
  \end{itemize}

 \item mycpとmycp2\_1の比較
 \begin{itemize}
   \item システム時間 \\
    システム時間は高水準I/Oを用いるmycpの方が圧倒的に短い.
   FILE構造体のバッファにデータをためてシステムコール回数を少なくしたお陰である.

  \item ユーザ時間 \\
  mycpはmycp2\_1と比較してユーザ時間が4分の1倍に短くなっている.
   \|read()|,\|write()|の呼出し回数と\|getc()|,\|putc()|の呼出し回数は
   同じはずである.
   \|getc()|,\|putc()|は,
   \|read()|,\|write()|関数中のシステムコールを呼び出す前処理
   (前処理までがuser時間に含まれる)より軽い.
   (これには驚いた.)
   macOSのオンラインマニュアル(\|$ man getc|)で調べてみると
   「\|getc()|はマクロでありインラインに展開される」と書いてある.
   \|getc()|,\|putc()|が,FILE構造体のバッファを操作する処理には
   高速化するための工夫が凝らされている.
  \end{itemize}

\end{enumerate}
\end{document}
