\documentclass[a4j,twcolumn,11pt,nomag]{ltjarticle}      % lualatex の場合
\usepackage{myAns}
\chead{\textgt{システムプログラミング2 課題 No.2 解答}}

\begin{document}
\onecolumn
\section*{課題 No.2 の解答例}

高水準I/Oを使用した場合,
バッファサイズの異なる低水準I/Oを使用した場合について,
ファイルをコピーする時間を比較した.

\subsection*{実行条件}
  \begin{center}
    \begin{tabular}{l c l}
      実行したコンピュータ & : & MacBook Pro M1 2020 \\
      実行コンピュータのOS & : & macOS 11.3 (Big Sur) \\
      ファイルサイズ      & : & 12MiB(bs=1024, count=12288) \\
    \end{tabular}  
  \end{center}  

\subsection*{実行結果}
実行時間は遅い方から以下の順であった.
\begin{enumerate}
\item 1バイトのwriteシステムコール(mycp2\_1)
  \begin{center}
    \begin{tabular}{|l | r | r | r | r |} \hline
      & \multicolumn{1}{|c|}{1回目}
      & \multicolumn{1}{|c|}{2回目}
      & \multicolumn{1}{|c|}{3回目}
      & \multicolumn{1}{|c|}{平均} \\\hline\hline
      real & 20.37 & 19.77 & 19.55 & 19.90 \\\hline
      user &  1.61 &  1.61 &  1.63 &  1.62 \\\hline
      sys  & 18.67 & 18.04 & 17.85 & 18.19 \\\hline
    \end{tabular}
  \end{center}
\item 高水準I/O(mycp)
  \begin{center}
    \begin{tabular}{|l | r | r | r | r |} \hline
      & \multicolumn{1}{|c|}{1回目}
      & \multicolumn{1}{|c|}{2回目}
      & \multicolumn{1}{|c|}{3回目}
      & \multicolumn{1}{|c|}{平均} \\\hline\hline
      real &  0.44 &  0.44 &  0.44 &  0.44 \\\hline
      user &  0.42 &  0.42 &  0.42 &  0.42 \\\hline
      sys  &  0.01 &  0.01 &  0.01 &  0.01 \\\hline
    \end{tabular}
  \end{center}
\item 1,024バイトのwriteシステムコール(mycp2\_1024)
  \begin{center}
    \begin{tabular}{|l | r | r | r | r |} \hline
      & \multicolumn{1}{|c|}{1回目}
      & \multicolumn{1}{|c|}{2回目}
      & \multicolumn{1}{|c|}{3回目}
      & \multicolumn{1}{|c|}{平均} \\\hline\hline
      real &  0.05 &  0.05 &  0.05 &  0.05 \\\hline
      user &  0.00 &  0.00 &  0.00 &  0.00 \\\hline
      sys  &  0.04 &  0.04 &  0.04 &  0.04 \\\hline
    \end{tabular}
  \end{center}
\end{enumerate}
\subsection*{考察}
\begin{enumerate}
  \item mycp2\_1はmycp2\_1024よりシステム時間が約400倍になっている.
   システムコールの呼び出し回数は約1000倍になっているはずなので,
   システムコール1回当たりの処理時間は0.4倍程度と考えられる.
   システムコール1回当たり読み書きするデータの量が1000倍になっても,
   システムコール1回当たりの処理時間は2.5倍(1/0.4倍)程度で済んでいる.
  \item mycpはmycp2\_1024と比較してシステム時間が短くなっている.
   このことから高水準I/Oが用いるバッファは1024バイトより大きく,
   mycpのシステムコール呼び出し回数は
   mycp2\_1024より少なくなっていると考えられる。
  \item mycpはmycp2\_1と比較してユーザ時間が4分の1倍に短くなっている.
   高水準I/Oの関数(\|getc()|,\|putc()|)は,
   システムコールを呼び出すための前処理より軽い.(これには驚いた.)
   オンラインマニュアル(\|man getc|)で調べてみると
   「\|getc()|はマクロでありインラインに展開される」と書いてある.
   高速化するための工夫が凝らされている.
\end{enumerate}
\end{document}
