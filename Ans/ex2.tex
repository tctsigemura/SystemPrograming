\documentclass[a4j,twcolumn,11pt,nomag]{ltjarticle}      % lualatex の場合
\usepackage{myAns}
\chead{\textgt{システムプログラミング2 課題 No.2 解答}}

\begin{document}
\onecolumn
\section*{課題 No.2 の解答例}
筆者が日頃使用している MacBook Pro 上で計測した結果を下の表にまとめた.
実行時間は遅い方から以下の順であった.

\begin{enumerate}
\item 1バイトのwriteシステムコール
\item 高水準I/O
\item 1,024バイトのwriteシステムコール
\end{enumerate}

カーネルが費やした時間は高水準I/Oが最も短いことから,
高水準I/Oは1,024バイトよりも大きいバッファを使用していると考えられる.

\begin{figure}[h]
  \begin{center}
  1バイトのwriteシステムコール使用\\
  \tbl{scale=0.9}{mycp2_1_10M-crop.pdf}
  \end{center}
\end{figure}

\begin{figure}[h]
  \begin{center}
    高水準I/O使用\\
    \tbl{scale=0.9}{mycp_10M-crop.pdf}
  \end{center}
\end{figure}

\begin{figure}[h]
  \begin{center}
    1,024バイトのwriteシステムコール使用\\
    \tbl{scale=0.9}{mycp2_1024_10M-crop.pdf}
  \end{center}
\end{figure}

\end{document}

