\documentclass[a4j,twcolumn,11pt,nomag]{ltjarticle}      % lualatex の場合
\usepackage{myAns}
\chead{\textgt{システムプログラミング2 課題 No.4 解答}}

\begin{document}
\onecolumn

\section*{課題 No.4 の解答例}
\begin{enumerate}
\item 簡易版のUNIXコマンドを作成しなさい.
\lst{numbers=none,language=C,caption=ファイルの削除(myrm.c)}{myrm.c}
\begin{lstlisting}[caption=myrmの実行例(動作テスト!!)]
$ echo aaa  > a.txt                                 <-- 実験用ファイルを作る
$ echo aaa  > b.txt
$ echo aaa  > c.txt
$ ls -l *.txt
-rw-r--r--  1 sigemura  staff  4 May 18 11:19 a.txt <-- できているか確認
-rw-r--r--  1 sigemura  staff  4 May 18 11:19 b.txt
-rw-r--r--  1 sigemura  staff  4 May 18 11:19 c.txt
$ ./myrm                                            <-- 引数が0個の場合
使用方法: ./myrm file ...
$ ./myrm /bin/rm                                    <-- システムのファイルは
/bin/rm: Permission denied                               消せない(権限無し)
$ ./myrm xyz                                        <-- 存在しないファイル
xyz: No such file or directory
$ ./myrm a.txt                                      <-- ファイルを1つ消す
$ ls -l *.txt
-rw-r--r--  1 sigemura  staff  4 May 18 11:19 b.txt <-- a.txt は消えた
-rw-r--r--  1 sigemura  staff  4 May 18 11:19 c.txt
$ ./myrm b.txt c.txt                                <-- まとめて2つ消す
$ ls -l *.txt
ls: *.txt: No such file or directory                <-- 両方消えた
$ 
\end{lstlisting}
%\newpage

\lst{caption=ディレクトリの作成(mymkdir.c)}{mymkdir.c}
\begin{lstlisting}[caption=mymkdirの実行例(動作テスト!!)]
$ ./mymkdir                                         <-- 引数がない場合
使用方法: ./mymkdir directory ...
$ ./mymkdir a                                       <-- 引数が1つの場合
$ ./mymkdir b c d                                   <-- 引数が3つの場合
$ ls -l a b c d
drwxr-xr-x  2 sigemura  kan   512 Jan 20 10:04 a
drwxr-xr-x  2 sigemura  kan   512 Jan 20 10:04 b
drwxr-xr-x  2 sigemura  kan   512 Jan 20 10:04 c
drwxr-xr-x  2 sigemura  kan   512 Jan 20 10:04 d
$ ./mymkdir a                                       <-- ディレクトリ名が衝突
a: File exists
$ ./mymkdir /a                                      <-- / には作れない
/a: Permission denied
$ ./mymkdir e a f                                   <-- 複数の1つがエラー
a: File exists
$ ls -l a b c d e f
drwxr-xr-x  2 sigemura  kan   512 Jan 20 10:04 a
drwxr-xr-x  2 sigemura  kan   512 Jan 20 10:04 b
drwxr-xr-x  2 sigemura  kan   512 Jan 20 10:04 c
drwxr-xr-x  2 sigemura  kan   512 Jan 20 10:04 d
drwxr-xr-x  2 sigemura  kan   512 Jan 20 10:05 e    <-- エラーにならなかった
drwxr-xr-x  2 sigemura  kan   512 Jan 20 10:05 f      ディレクトリはできてる
$
\end{lstlisting}
%\newpage

\lst{caption=ディレクトリの削除(myrmdir.c)}{myrmdir.c}
\begin{lstlisting}[caption=myrmdirの実行例(動作テスト!!)]
$ ./myrmdir                                     <-- 引数がない場合
使用方法: ./myrmdir directory ...
$ ./myrmdir a                                   <-- 引数が1つの場合
$ ./myrmdir b c d                               <-- 引数が3つの場合
$ ./myrmdir a                                   <-- ディレクトリが存在しない
a: No such file or directory
$ ./myrmdir myrmdir.c                           <-- ディレクトリではない
myrmdir.c: Not a directory
$ ./myrmdir /tmp                                <-- システムのディレクトリ
/tmp: Permission denied
$ ls -l                                         <-- 結果の確認
total 38
...
drwxr-xr-x  2 sigemura  kan   512 Jan 20 10:05 e
drwxr-xr-x  2 sigemura  kan   512 Jan 20 10:05 f
-rwxr-xr-x  1 sigemura  kan  1234 Jan 20 10:05 myrmdir.c
...
$ ./myrmdir ?                                   <-- ワイルドカード
$ ls -l                                         <-- 結果の確認
...                                             <-- e,fが消えた
-rwxr-xr-x  1 sigemura  kan  1234 Jan 20 10:05 myrmdir.c
...
$
\end{lstlisting}
%\newpage

\lst{caption=ハードリンクの作成(myln.c)}{myln.c}
\begin{lstlisting}[caption=mylnの実行例(動作テスト!!)]
$ ./myln                                            <- 引数の数が 0 個
使用方法: ./myln file1 file2
$ ./myln a                                          <- 引数の数が 1 個
使用方法: ./myln file1 file2
$ ./myln a b c                                      <- 引数の数が 3 個
使用方法: ./myln file1 file2
$ echo aaaa > a                                     <- ファイル a を作る
$ ls -l                                             <- a ができたか確認
total 36
-rw-r--r--  1 sigemura  kan     5 May 10 14:12 a
$ ./myln a b                                        <- a を b へハードリンク
$ ls -l                                             <- b ができたか確認
total 40
-rw-r--r--  2 sigemura  kan     5 May 10 14:12 a
-rw-r--r--  2 sigemura  kan     5 May 10 14:12 b
$ ./myln a b
link: File exists                                   <- 既にある名前を使うと
$ ./myln x y
link: No such file or directory                     <- 存在しない名前を使うと
$
\end{lstlisting}
\newpage

%\lst{caption=シンボリックリンクの作成(mylns.c)}{mylns.c}
% \begin{lstlisting}[caption=mylnsの実行例(動作テスト!!)]
% $ ./mylns
% 使用方法: ./mylns file1 file2
% $ ./mylns b.txt
% 使用方法: ./mylns file1 file2
% $ ./mylns b.txt a.txt
% $ echo abc > b.txt
% $ ls -l
% total 10
% lrwxr-xr-x  1 sigemura  staff   5 Apr 28 17:03 a.txt -> b.txt
% -rw-r--r--  1 sigemura  staff   4 Apr 28 17:05 b.txt
% $ cat a.txt
% abc
% $ ./mylns b.txt a.txt                       <- a.txt はすでに存在
% a.txt: File exists
% $ ./mylns zzz.txt c.txt                     <- リンク先は存在しなくてもよい
% $ 
% \end{lstlisting}
%\newpage

\lst{caption=ファイルの移動(名前変更)(mymv.c)}{mymv.c}
\begin{lstlisting}[caption=mymvの実行例(動作テスト!!)]
$ ./mymv
使用方法: ./mymv <もとの名前> <新しい名前>
$ ls -l *.txt
-rw-r--r--  1 sigemura  staff  1536 Feb  9 11:06 s.txt
$ ./mymv s.txt x.txt                <-- x.txt に変更した
$ ls -l *.txt
-rw-r--r--  1 sigemura  staff  1536 Feb  9 11:06 x.txt
$ ./mymv s.txt x.txt                <-- s.txt が存在しない場合
./mymv: No such file or directory
$ ./mymv x.txt /x.txt               <-- / に移動する権限がない
./mymv: Permission denied
$ ./mymv x.txt /tmp/x.txt           <-- ハードディスクをまたいだ移動はできない
./mymv: Cross-device link
$ 
\end{lstlisting}
\newpage

% \lst{caption=ファイルのモード変更(mychmod.c)}{mychmod.c}
% \begin{lstlisting}[caption=mychmodの実行例(動作テスト!!)]
% $ ls -l a b
% -rw-r--r--  1 sigemura  kan  0 May  6 21:58 a
% -rw-r--r--  1 sigemura  kan  0 May  6 21:58 b
% $ ./mychmod
% 使用方法 : ./mychmod <8進数> <file> [<file>...]
% $ ./mychmod 755 a b
% $ ls -l a b
% -rwxr-xr-x  1 sigemura  kan  0 May  6 21:58 a
% -rwxr-xr-x  1 sigemura  kan  0 May  6 21:58 b
% $ ./mychmod 644 b
% $ ls -l a b
% -rwxr-xr-x  1 sigemura  kan  0 May  6 21:58 a
% -rw-r--r--  1 sigemura  kan  0 May  6 21:58 b
% $ ./mychmod 644 c
% c: No such file or directory
% $ ./mychmod 789 a
% '789' : 8進数の形式が不正
% $ ./mychmod 12345 a
% '12345' : 8進数の値が不正
% $
% \end{lstlisting}

\item 本物のUNIXコマンドとの相違

\begin{enumerate}
\item 複数のファイルを一度に操作できる。

上の解答では複数のファイルを一度に操作できるプログラムになっている。
皆さんが作ったプログラムは、ほとんどが一度に一つのファイルしか操作できない。

\item 色々なオプションが用意されている。

\begin{lstlisting}[caption=rm コマンドのマニュアル(一部)]
NAME
     rm, unlink -- remove directory entries

SYNOPSIS
     rm [-dfiPRrvW] file ...
     unlink file

DESCRIPTION
...
\end{lstlisting}

\item 以下は間違え

\begin{enumerate}
\item 拡張子の操作関係

拡張子は単にファイル名の一部である。
特別扱いは必要ない。

\item ワイルドカード関連

ワイルドカードはシェルの機能である。
ワイルドカードをシェルが展開した後、
展開結果を用いてコマンドが起動される。(myrmdir の実行例参照)

\end{enumerate}
\end{enumerate}

\item rename システムコールの必要性
%最初の二つは、UNIXファイルシステムだけを使用する場合でも問題になる。
%最後の一つは、色々なファイルシステムが混在する現代のシステムで問題になる。
%(現代のシステムは、ハードディスクはUNIXファイルシステム、
%USBメモリはFATファイルシステム、
%ネットワークを用いたファイル共有はWindowsのファイル共有方式
%を使用する等、色々なファイルシステムが混在している。
%ファイルシステムの構造に依存する操作は好ましくない。)

\begin{enumerate}
\item ディレクトリの移動が link-unlink ではできない。

ディレクトリの link は禁止されている。
(ディレクトリの link を認めるとファイル木が複雑になりすぎる。)
ディレクトリの移動が可能なシステムコールが必要である。

\item ファイルの移動はアトミック操作であるべき。

link-unlink手法は複数のステップで目的を完了するので、
何らかの理由(例えばユーザのCtrl-C)でプログラムが途中で終了したら、
link は完了したが unlink をする前の中途半端な状態になる可能性がある。
そのような状態になるのを防ぐために、
アトミック操作(不可分操作)としてファイル移動操作を行う
rename システムコールが提供される。

通常、システムコールの途中でプログラムは終了できないので、
一つのシステムコールで全ての操作を行えば途中の状態で終わることがない。

\item ハードリンクが使用できないファイルシステムでもファイルの移動は必要。

例えば、IE電算のMacのホームディレクトリ(ネットワークドライブ)では
ハードリンクが使用できない。
(使用できないから\texttt{/tmp}でハードリンクの演習をした。)
その他に、FATファイルシステム(USBメモリ等)でもハードリンクが使用できない。

ハードリンクが使用できないファイルシステムでは、
link-unlinkの手法は使用できない。
ハードリンクが使用できるかできないかに関係なく、
ファイルの移動を同じ手法(システムコール)で行えるべきである。
\end{enumerate}

% \item stat システムコール,readdir 関数について調べる.
% \begin{enumerate}
% \item stat システムコール\\
% ファイルの属性情報(メタデータ)を読み出すシステムコールである.
% \item readdir 関数\\
% opendir関数で開いてあるディレクトリファイルから,
% ディレクトリエントリを読み出す関数である.
% \end{enumerate}

% ls コマンドでは,これらを組み合わせて使っている筈だ.
% (ファイルの一覧を知るためにreaddir関数,
% \texttt{ls -l}で属性情報を表示するためにstatシステムコールが必要だ.)
\end{enumerate}
\end{document}

