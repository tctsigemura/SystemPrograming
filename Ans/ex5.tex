\documentclass[a4j,twcolumn,11pt,nomag]{ltjarticle}      % lualatex の場合
\usepackage{myAns}
\chead{\textgt{システムプログラミング2 課題 No.5 解答}}

\begin{document}
\onecolumn

\section*{課題 No.5 の解答例}
\begin{itemize}
\item[課題5-1]
  プロセス数を調べる.
  \begin{enumerate}
  \item[a.] システム内の全プロセス数を求めなさい.
\begin{lstlisting}
psオプション
 a:他人のプロセスも表示
 u:詳細な情報を表示
 x:端末に属さないプロセスも表示
$ ps ax | wc
     536    3228   75717
psコマンドの出力には1行のヘッダが付くので 536-1=535プロセス
\end{lstlisting}
  \item[b.] システム内の全自プロセス数を求めなさい.
\begin{lstlisting}
psが全自分プロセスを表示する
$ ps x | wc
     317    2018   55223
psコマンドの出力には1行のヘッダが付くので 317-1=316プロセス
\end{lstlisting}
\begin{lstlisting}
psはロングフォーマットでシステムの全プロセスを出力し,
例えばsigemuraで始まる行だけを選びカウントする.
$ ps aux | grep ^sigemura | wc
     314    3883   71726
grepプロセス分多いはずなのに逆に2少ない.
ps x ではrootのloginプロセスが3つ含まれているたが今回は含まれない.
\end{lstlisting}
  \end{enumerate}

\item[課題5-2]
二つのターミナルを用いてタイマープログラムプロセスを観察する.
  \begin{enumerate}
  \item タイマープログラムのPIDとSTATの確認
\lstset{basicstyle=\footnotesize\ttfamily}
\begin{lstlisting}
% ps u
USER       PID  %CPU %MEM      VSZ    RSS   TT  STAT STARTED      TIME COMMAND
sigemura 17419   0.0  0.0 407972224   1296 s001  S+   11:04AM   0:00.01 ./timer
sigemura 17367   0.0  0.0 408300576   3840 s001  S    11:02AM   0:00.09 -zsh
sigemura 17026   0.0  0.0 408169504   3616 s000  S    10:28AM   0:00.07 -zsh
\end{lstlisting}
\|timer|のPIDは17419,STATは\|S+|(短時間のスリープ)

  \item タイマープログラムを一時停止させる.
\lstset{basicstyle=\small\ttfamily}
\begin{lstlisting}
% kill -TSTP 17419
\end{lstlisting}
もう一方のターミナルの時計表示が止まった.

  \item タイマープログラムのSTATを再度確認
\lstset{basicstyle=\footnotesize\ttfamily}
\begin{lstlisting}
% ps u
USER       PID  %CPU %MEM      VSZ    RSS   TT  STAT STARTED      TIME COMMAND
sigemura 17026   0.3  0.0 408169504   3376 s000  S    10:28AM   0:00.11 -zsh
sigemura 17419   0.0  0.0 407972224    816 s001  T    11:04AM   0:00.10 ./timer
sigemura 17367   0.0  0.0 408300576   3376 s001  S+   11:02AM   0:00.09 -zsh
\end{lstlisting}
STATが\|T|(一時停止状態)に変化した.

  \item タイマープログラムを再開させる.
\lstset{basicstyle=\small\ttfamily}
\begin{lstlisting}
% kill -CONT 17419
\end{lstlisting}
もう一方のターミナルの時計表示が再開された.

  \item タイマープログラムを終了させる.
\begin{lstlisting}
% kill -INT 17419
\end{lstlisting}
「課題の準備」でタイマープログラムは\|^C|で画面を消して終了していた.
\|^C|と同じシグナルを送ると同じ終わり方をする.

  \end{enumerate}

\item[課題5-3] 以下を一つのターミナルで行う.
  \begin{enumerate}
  \item タイマープログラムを(起動した後,)一時停止する.
\begin{lstlisting}
% ./timer
...時計の表示がされる...
^Z
zsh: suspended  ./timer
% 
\end{lstlisting}
\|^Z|を入力すると一時停止しプロンプトが表示される.

  \item 一時停止したタイマープログラムを再開する.
\begin{lstlisting}
% fg
...時計の表示が再開される...
\end{lstlisting}
\|fg|と入力するとプログラムがフォアグラウンドで実行再開する.

  \item タイマープログラムを終了させる.
\begin{lstlisting}
...時計の表示がされている...
^C
...画面がクリアされる...
%
\end{lstlisting}
\|^C|を入力するとプログラムが終了しプロンプトが表示される.

  \end{enumerate}
\end{itemize}
\end{document}
