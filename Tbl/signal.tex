\RequirePackage{luatex85}
\documentclass[border=1mm]{standalone}
\usepackage{luatexja}                              % lualatex の場合
\usepackage[hiragino-pron]{luatexja-preset}        % ヒラギノフォント
\newcommand{\bs}{\texttt{\char'134}}  % バックスラッシュを表示するコマンド
\def\|{\verb|} %|
\begin{document}
\begin{tabular}{r | l | l | l}\hline\hline
\multicolumn{1}{c|}{番号} &
\multicolumn{1}{c|}{記号名} &
\multicolumn{1}{c|}{デフォルト} &
\multicolumn{1}{c}{説明} \\\hline
1  & \|SIGHUP|  & 終了       & プロセスが終了していないときログアウトした. \\
2  & \|SIGINT|  & 終了       & ターミナルでCtrl-Cが押された. \\
3  & \|SIGQUIT| & コアダンプ & ターミナルでCtrl-{\bs}が押された. \\
4  & \|SIGILL|  & コアダンプ & 不正な機械語命令を実行した. \\
8  & \|SIGFPE|  & コアダンプ & 演算でエラーが発生した. \\
9  & \|SIGKILL| & 終了   & 強制終了(\emph{ハンドリングの変更ができない}). \\
10 & \|SIGBUS|  & コアダンプ & 不正なアドレスをアクセスした場合など.\\
11 & \|SIGSEG|  & コアダンプ & 不正なアドレスをアクセスした場合など.\\
14 & \|SIGALRM| & 終了       & \|alarm()|で指定した時間が経過した. \\
15 & \|SIGTERM| & 終了       & 終了. \\
17 & \|SIGSTOP| & 停止   & 一時停止(\emph{ハンドリングの変更ができない}). \\
18 & \|SIGTSTP| & 停止       & ターミナルでCtrl-Zが押された. \\
19 & \|SIGCONT| & 無視       & 一時停止中なら再開する. \\
20 & \|SIGCHLD| & 無視       & 子プロセスの状態が変化した. \\
\end{tabular}
\end{document}
